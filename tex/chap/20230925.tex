\chapter{2023/9/25}\label{20230925}

\section{Gravity and gravitational potential 引力和引力势}\label{gravity-and-gravitational-potential-ux5f15ux529bux548cux5f15ux529bux52bf}

The gravitational potential at \(r\) is \[U = - {GMm \over r},\] and the force there is \[\boldsymbol F(\boldsymbol r) = - \nabla U = - {GMm \over r^2} \hat {\boldsymbol r}.\]

\emph{为什么有\(r^2\)项?因为场是点源产生的球面场,场对应的力有\(r^2\)形式。}

We can integrate \(\boldsymbol F(\boldsymbol r)\) to get \(U(r_0)\): \[U(r_0) = \int^{+\infty}_{r_0} -{GMm \over r^2} \mathrm dr = \left.{GMm \over r} \right\vert^{+\infty}_{r_0} = -{GMm \over r_0}.\]

Near the ground of earth, at height \(h(h \ll R)\), we can have approximately \(U=mgh\). That's because \[U(R+h)- U(R)  = {GMm \over R} - {GMm \over R+h} = GMm \cdot {h \over R(R+h)} \approx {GMmh \over r^2} = mgh.\]

\section{The Great Debate for String, 1730\textasciitilde1780}\label{the-great-debate-for-string-17301780}

\begin{itemize}
\tightlist{}
\item
  Daniel Bernoulli: the first to propose a partial differential equation
  第一个提出偏微分方程的人
\item
  D'Alembert: superposition principle 叠加原理
\item
  Lagrange
\item
  Euler
\end{itemize}

John Bernoulli: the discretization of string vibration (弦振动的离散化), which sees the string as a string of beads (珠子).

\section{Gauss's theorem (electromagnetics) 高斯定理}\label{gausss-theorem-electromagnetics-ux9ad8ux65afux5b9aux7406}

\emph{The three most renowned mathematicians: Archimedes (阿基米德), Isaac Newton (伊萨克·牛顿), and \textbf{Gauss} (高斯).}

Differential form (微分形式): \[\nabla \cdot \boldsymbol E = {\rho \over \varepsilon_0}, \] or integral form (积分形式): \[\oiint_{\partial V} \boldsymbol E \cdot \mathrm d \boldsymbol S = {Q_{\mathrm{enc}} \over \varepsilon_0},\] in which ``\(\mathrm {enc}\)'' stands for \textbf{enclosed}.
