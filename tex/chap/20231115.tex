\chapter{2023/11/15}\label{20231115}

\section{Mechanical similarity
力学相对性}\label{mechanical-similarity-ux529bux5b66ux76f8ux5bf9ux6027}

From \emph{Mathematical Method in Classical Mechanics} by Vladimir
Igorevich Arnold (Russian: Влади́мир И́горевич Арно́льд,
中文:弗拉基米尔·伊戈列维奇·阿诺尔德).

\emph{上面这个人在19岁(1957年)时解决了希尔伯特第十三问题,名满天下!要向他学习!}

\begin{quote}
作业1:什么是希尔伯特第十三问题?是何时解决的?

答案:希尔伯特第十三问题是:\(f^7 + x f^3 +y f^2 + zf + 1 = 0\)
这个方程式的七个解\(f\),若表成\(x, y, z\)的函数,是否可以化成两个变量函数的组合
\(f \circ g\)。

1957年,阿诺尔德解决了这个问题,答案是``是''。
\end{quote}

\begin{quote}
作业2:为什么所有动物跳的高度都是同一量级?
\end{quote}

Assume that a potential energy in a system
\(U(\boldsymbol{r}_1, \boldsymbol{r}_2, \dots, \boldsymbol{r}_n)\) is a
homogeneous function of degree \(k\). This means that
\[U(\alpha \boldsymbol{r}_1, \alpha \boldsymbol{r}_2, \dots, \alpha \boldsymbol{r}_n) = \alpha^k U(\boldsymbol{r}_1, \boldsymbol{r}_2, \dots, \boldsymbol{r}_n).\]

Thus the Lagrangian (拉格朗日量) \(L\) is
\[L = T - U = \sum_{n} {1 \over 2} m {\mathrm{d} \boldsymbol{r}_i \over \mathrm{d} t} \cdot {\mathrm{d} \boldsymbol{r}_i \over \mathrm{d} t} - U(\boldsymbol{r}_1, \boldsymbol{r}_2, \dots, \boldsymbol{r}_n).\]

\begin{quote}
作业3:为什么拉格朗日量 \(L = T - U\)?
\end{quote}

Let \(\boldsymbol{r}_1' = \alpha \boldsymbol{r}_1\), \(t' = \beta t\),
and
\[L' = T' - U' = \left( {\alpha \over \beta} \right)^2 T - \alpha^k U.\]

To let the Euler--Lagrange equation
\[\frac{\partial L}{\partial q} - \frac{\mathrm{d}}{\mathrm{d}t}\frac{\partial L}{\partial \dot q} = 0\]
still hold, we need \[\left( {\alpha \over \beta} \right)^2 = \alpha^k\]
or \[{t' \over t} = \left( {l' \over l} \right)^{1 - k/2}.\]

Following this we can know some other physical quantities, for example:
\[{E' \over E} = {U' \over U} = \alpha^k;\]
\[{\boldsymbol{v}' \over \boldsymbol{v}} = {l'/t' \over l/t} = {\alpha \over \beta} = \alpha^{k/2};\]
\[{\boldsymbol{L}' \over \boldsymbol{L}} = {mv'l' \over mvl} = {\alpha^2 \over \beta} = \alpha^{1 + k/2}.\]

Some applications of this include:

\begin{enumerate}
\def\labelenumi{\arabic{enumi}.}
\item
  In the universal gravity field, \(k = -1\), and
  \[{t' \over t} = \left( {l' \over l} \right)^{3/2},\] which is in
  accordance with \textbf{Kepler's Third Law (开普勒第三定律)}.
\item
  For a string, \(k = 2\), and
  \[{t' \over t} = \left( {l' \over l} \right)^{0} = 1,\] which means
  that the period \(T\) is independent of the amplitude of the string
  \(A\).
\end{enumerate}

\begin{quote}
作业4:

\begin{enumerate}
\def\labelenumi{\arabic{enumi}.}
\item
  质量不同势能相同的质点沿着相同轨道运动,它们的运动时间满足什么关系?

  答案:\[{t' \over t} = \sqrt{m' \over m}\]
\item
  质点有相同的质量但势能相差一个常数因子,试求沿着相同轨道运动的时间之比?

  答案:\[{t' \over t} = \sqrt{U \over U'}\]
\end{enumerate}
\end{quote}

\begin{quote}
作业5:使用量纲分析多米诺骨牌倒下用时与相关物理量的关系
\end{quote}

From Newton's second law of motion, we know that
\[\boldsymbol{F} = \dfrac{\mathrm{d} \boldsymbol{p}}{\mathrm{d}t}.\]

Under a potential field \(U\), we have
\[\boldsymbol{F} = - {\partial U \over \partial q}\] and
\[\boldsymbol{p} = {\partial T \over \partial \dot{q}},\] and
consequently,
\[- {\partial U \over \partial q} = \dfrac{\mathrm{d}}{\mathrm{d}t} {\partial T \over \partial \dot{q}}.\]

Because \(U = U(q_1, q_2, \dots, q_n)\) and
\(T = T(\dot{q}_1, \dot{q}_2, \dots, \dot{q}_n)\), we have
\[{\partial U \over \partial \dot{q}} = 0\] and
\[{\partial T \over \partial q} = 0.\]

From these we can have
\[{\partial (T - U) \over \partial q} - \dfrac{\mathrm{d}}{\mathrm{d}t} {\partial (T - U) \over \partial \dot{q}} = 0,\]
\[{\partial L \over \partial q} - \dfrac{\mathrm{d}}{\mathrm{d}t} {\partial L \over \partial \dot{q}} = 0.\]

\section{Euler's equations in rigid body dynamics
刚体运动的欧拉方程}\label{eulers-equations-in-rigid-body-dynamics-ux521aux4f53ux8fd0ux52a8ux7684ux6b27ux62c9ux65b9ux7a0b}

\begin{quote}
作业6:拉格朗日量 \(L\) 为什么不写成 \(\ddot q\) 的函数?
\end{quote}

According to the angular momentum theorem (角动量定理), we have
\[\dfrac{\mathrm d \boldsymbol L}{\mathrm dt} = \boldsymbol M,\] where
\(\boldsymbol M\) is the external torque (外力矩) and
\(\boldsymbol L = \mathbf I \boldsymbol \omega\).

According to the transport theorem,
\[\dfrac{\mathrm d \boldsymbol L}{\mathrm dt} = \left[ \dfrac{\mathrm d (\mathbf I \boldsymbol \omega)}{\mathrm dt}\right]_r + \boldsymbol \omega \times (\mathbf I \boldsymbol \omega) = \boldsymbol M.\]

Because \(\mathbf I\) is a constant, we have
\[\mathbf I \left( \dfrac{\mathrm d \boldsymbol \omega}{\mathrm dt} \right)_r + \begin{vmatrix}
    \hat i & \hat j & \hat k \\
    \omega_1 & \omega_2 & \omega_3 \\
    \mathrm I_{11} \omega_1 & \mathrm I_{22} \omega_2 & \mathrm I_{33} \omega_3 \\
\end{vmatrix} =
\begin{pmatrix}
    M_1 \\ M_2 \\ M_3
\end{pmatrix}.\]

In orthogonal principal axes of inertia coordinates, the equations
become \[\left\{
    \begin{aligned}
        \mathrm I_{11} \dfrac{\mathrm d \boldsymbol \omega_1}{\mathrm dt} + (\mathrm I_{33} - \mathrm I_{22}) \omega_2 \omega_3 & = M_1; \\
        \mathrm I_{22} \dfrac{\mathrm d \boldsymbol \omega_2}{\mathrm dt} + (\mathrm I_{11} - \mathrm I_{33}) \omega_1 \omega_3 & = M_2; \\
        \mathrm I_{33} \dfrac{\mathrm d \boldsymbol \omega_3}{\mathrm dt} + (\mathrm I_{22} - \mathrm I_{11}) \omega_1 \omega_2 & = M_3.
    \end{aligned}
\right.\]

\section{Intermediate axis theorem
中间轴定理}\label{intermediate-axis-theorem-ux4e2dux95f4ux8f74ux5b9aux7406}

\begin{quote}
作业7:查一下礼炮七号发生的事故
\end{quote}

\emph{酒泉卫星发射中心:向航天强国奋力迈进}

The intermediate axis theorem is also called the tennis racket theorem
(网球拍定理) or Dzhanibekov effect (贾尼别科夫效应). Vladimir
Aleksandrovich Dzhanibekov (Russian: Владимир Александрович Джанибеков,
中文:弗拉基米尔·亚历山德罗维奇·贾尼别科夫) noticed a logical
consequence of this theorem in space in 1985.

Now we assume that, for a rigid body,
\(\mathrm I_{11} > \mathrm I_{22} > \mathrm I_{33}\), and:

\begin{itemize}
\tightlist{}
\item
  Case 1: rotation around \(\mathrm I_{11}\)

  Here we assume that \(\omega_1 = \Omega\), and
  \(\omega_2, \omega_3 \ll \Omega\), and according to the angular
  momentum theorem, we have \[\left\{
        \begin{aligned}
            \mathrm I_{11} \dfrac{\mathrm d \boldsymbol \omega_1}{\mathrm dt} & = (\mathrm I_{22} - \mathrm I_{33}) \omega_2 \omega_3 \approx 0; \\
            \mathrm I_{22} \dfrac{\mathrm d \boldsymbol \omega_2}{\mathrm dt} & = (\mathrm I_{33} - \mathrm I_{11}) \Omega \omega_3; \\
            \mathrm I_{33} \dfrac{\mathrm d \boldsymbol \omega_3}{\mathrm dt} & = (\mathrm I_{11} - \mathrm I_{22}) \Omega \omega_2.
        \end{aligned}
    \right.\]

  Take the time derivatives of these equations, and we get \[\left\{
        \begin{aligned}
            \mathrm I_{11} \dfrac{\mathrm d^2 \boldsymbol \omega_1}{\mathrm dt^2} & = 0; \\
            \mathrm I_{22} \dfrac{\mathrm d^2 \boldsymbol \omega_2}{\mathrm dt^2} & = (\mathrm I_{33} - \mathrm I_{11}) \Omega \dfrac{\mathrm d \boldsymbol \omega_3}{\mathrm dt} = \dfrac{(\mathrm I_{33} - \mathrm I_{11})(\mathrm I_{11} - \mathrm I_{22})}{\mathrm I_{33}} \Omega^2 \omega_2; \\
            \mathrm I_{33} \dfrac{\mathrm d^2 \boldsymbol \omega_3}{\mathrm dt^2} & = (\mathrm I_{11} - \mathrm I_{22}) \Omega \dfrac{\mathrm d \boldsymbol \omega_2}{\mathrm dt} = \dfrac{(\mathrm I_{33} - \mathrm I_{11})(\mathrm I_{11} - \mathrm I_{22})}{\mathrm I_{33}} \Omega^2 \omega_3.
        \end{aligned}
    \right.\]

  Here, we have (\(\omega_3\) is similar to \(\omega_2\))
  \[\ddot{\omega}_2 + \dfrac{(\mathrm I_{11} - \mathrm I_{33})(\mathrm I_{11} - \mathrm I_{22})}{\mathrm I_{22} \mathrm I_{33}} \Omega^2 \omega_2 = 0.\]

  Let \(\dfrac{(\mathrm I_{11} - \mathrm I_{33})(\mathrm I_{11} - \mathrm I_{22})}{\mathrm I_{22} \mathrm I_{33}} \Omega^2= \alpha_2\). Because \(\alpha_2 > 0\), the general solution is \[\omega_2 = C_1 \sin \left( \sqrt{\alpha_2}t \right) + C_2 \cos \left( \sqrt{\alpha_2}t \right).\]

  This is a periodic, bounded function (周期性有界函数), which means
  that the rotation of directions \(2\) and \(3\) can be kept within a
  certain range.
\item
  Case 2: rotation around \(\mathrm I_{22}\)

  Here we assume that \(\omega_2 = \Omega\), and
  \(\omega_1, \omega_3 \ll \Omega\).

  Similar to Case 1, we know that
  \[\ddot{\omega}_1 = \dfrac{(\mathrm I_{11} - \mathrm I_{22})(\mathrm I_{22} - \mathrm I_{33})}{\mathrm I_{11} \mathrm I_{33}} \Omega^2 \omega_1.\]

  Let
  \(\dfrac{(\mathrm I_{11} - \mathrm I_{22})(\mathrm I_{22} - \mathrm I_{33})}{\mathrm I_{11} \mathrm I_{33}} \Omega^2 = \alpha_1\).

  Because \(\alpha_1 > 0\), the general solution is
  \[\omega_1 = C_1 \exp \left( \sqrt{\alpha_1} t \right) + C_2 \exp \left( - \sqrt{\alpha_1} t \right).\]

  This is not a periodic, bounded function, which means that the
  rotation of direction \(1\) can be kept within a certain range.
\end{itemize}
