\chapter{2023/10/18}\label{20231018}

\section{Warming Up}\label{warming-up}

\subsection*{(1) In quantum mechanics:}\label{in-quantum-mechanics}

\[\hat {\boldsymbol L} = \boldsymbol r \times \hat {\boldsymbol p} = - \mathrm i \hbar \boldsymbol r \times \nabla = - \mathrm i \hbar (x_i \boldsymbol e_i) \times \left({\partial \over \partial x_j }\boldsymbol e_j\right)= - \mathrm i \hbar x_i {\partial \over \partial x_j } \varepsilon_{ijk} \boldsymbol e_k.\]

\subsection*{(2) Coriolis acceleration
科氏加速度}\label{coriolis-acceleration-ux79d1ux6c0fux52a0ux901fux5ea6}

\[a_\text{Cor} = 2 \boldsymbol \omega \times \boldsymbol v' = 2 \omega_i v_j' \varepsilon_{ijk} \boldsymbol e_k.\]

\section{Matrices and determinants
矩阵和行列式}\label{matrices-and-determinants-ux77e9ux9635ux548cux884cux5217ux5f0f}

\emph{(译者注:matrix的复数形式是matrices)}

Define matrix \(\mathbf A\): \[\mathbf A=
\begin{bmatrix}
    a_{11} & a_{12} & a_{13} \\
    a_{21} & a_{22} & a_{23} \\
    a_{31} & a_{32} & a_{33} \\
\end{bmatrix},\] where \(a_{rc}\) refers to the entry located in row
\(r\) and column \(c\).

For example, We can write the Kronecker delta in matrix form:
\[\delta_{ij} = \begin{bmatrix}
    1 & 0 & 0 & \ldots & 0 \\
    0 & 1 & 0 & \ldots & 0 \\
    0 & 0 & 1 & \ldots & 0 \\
    \vdots & \vdots & \vdots & \ddots & \vdots \\
    0 & 0 & 0 & \ldots & 1
\end{bmatrix} = \mathbf{I}.\]

And determinant (行列式) of \(\mathbf A\): \begin{align*}
    \det (\mathbf A) & =
    \begin{vmatrix}
        a_{11} & a_{12} & a_{13} \\
        a_{21} & a_{22} & a_{23} \\
        a_{31} & a_{32} & a_{33} \\
    \end{vmatrix} \\
    & = 
    a_{11} \begin{vmatrix}
        a_{22} & a_{23} \\
        a_{32} & a_{33} \\
    \end{vmatrix} + 
    a_{12} \begin{vmatrix}
        a_{23} & a_{21} \\
        a_{33} & a_{31} \\
    \end{vmatrix} + 
    a_{13} \begin{vmatrix}
        a_{21} & a_{22} \\
        a_{31} & a_{32} \\
    \end{vmatrix} \\
    & = a_{11}(a_{22}a_{33} - a_{23}a_{32}) + a_{12}(a_{23}a_{31} - a_{21}a_{33}) + a_{13}(a_{21}a_{32} - a_{22}a_{31} ) \\
    & = (a_{11}a_{22}a_{33} + a_{12}a_{23}a_{31} + a_{13}a_{21}a_{32}) - (a_{11}a_{23}a_{32} + a_{12}a_{21}a_{33} + a_{13}a_{22}a_{31}). \quad (*)
\end{align*}

In \((*)\), you can see that the first digit follows the even
permutation, and the second digit follows the even permutation (for
positive terms) and odd permutation (for negative terms).

We can put \(\det \mathbf A\) in \textbf{compact/component form}
(紧凑/指标形式):
\[\det(\mathbf{A}) = \varepsilon_{ijk} a_{1i} a_{2j} a_{3k} = \varepsilon_{\alpha \beta \gamma} a_{\alpha 1} a_{\beta 2} a_{\gamma 3}.\]

\section{The product of two Levi-Civita symbols
两个列维-奇维塔符号的乘积}\label{the-product-of-two-levi-civita-symbols-ux4e24ux4e2aux5217ux7ef4-ux5947ux7ef4ux5854ux7b26ux53f7ux7684ux4e58ux79ef}

The default situation, or one with no repeated indices is this:
\begin{align*}
    \varepsilon_{ijk}\varepsilon_{lmn} &= \begin{vmatrix}
        \delta_{i\ell} & \delta_{im} & \delta_{in} \\
        \delta_{j\ell} & \delta_{jm} & \delta_{jn} \\
        \delta_{k\ell} & \delta_{km} & \delta_{kn} \\
    \end{vmatrix} \\
    &= \delta_{i\ell}\left( \delta_{jm}\delta_{kn} - \delta_{jn}\delta_{km}\right) - \delta_{im}\left( \delta_{j\ell}\delta_{kn} - \delta_{jn}\delta_{k\ell} \right) + \delta_{in} \left( \delta_{j\ell}\delta_{km} - \delta_{jm}\delta_{k\ell} \right).
\end{align*}

Below, without special mentioning, we default to 3-dimensional
conditions.

\subsection*{(1) One repeated index
1个指标重复}\label{one-repeated-index-1ux4e2aux6307ux6807ux91cdux590d}

Let \(i=\ell\), and \begin{align*}
    \varepsilon_{ijk}\varepsilon^{imn} &= \begin{vmatrix}
        {\delta_i}^i & {\delta_i}^m & {\delta_i}^n \\
        {\delta_j}^i & {\delta_j}^m & {\delta_j}^n \\
        {\delta_k}^i & {\delta_k}^m & {\delta_k}^n \\
    \end{vmatrix} \\
    &= {\delta_i}^i \left( {\delta_j}^m {\delta_k}^n - {\delta_j}^n {\delta_k}^m \right) - {\delta_i}^m \left( {\delta_j}^i {\delta_k}^n - {\delta_j}^n {\delta_k}^i \right) + {\delta_i}^n \left( {\delta_j}^i {\delta_k}^m - {\delta_j}^m {\delta_k}^i \right) \\
    &= 3\left( {\delta_j}^m {\delta_k}^n - {\delta_j}^n {\delta_k}^m \right) - ({\delta_i}^m {\delta_j}^i) {\delta_k}^n + ({\delta_i}^m {\delta_k}^i) {\delta_j}^n + ({\delta_i}^n {\delta_j}^i) {\delta_k}^m - ({\delta_i}^n {\delta_k}^i) {\delta_j}^m \\
    &= 3\left( {\delta_j}^m {\delta_k}^n - {\delta_j}^n {\delta_k}^m \right) - {\delta_j}^m {\delta_k}^n + {\delta_k}^m {\delta_j}^n + {\delta_j}^n {\delta_k}^m - {\delta_k}^n {\delta_j}^m \\ 
    &= {\delta_j}^m {\delta_k}^n - {\delta_j}^n {\delta_k}^m.
\end{align*}

\emph{助记:正序 \(-\) 逆序}

Example: \begin{align*}
    \boldsymbol a \times \left(\boldsymbol b \times \boldsymbol c\right)
    & = \boldsymbol a \times \left(b_i\boldsymbol e_i \times c_j \boldsymbol e_j \right) \\
    & = \left(a_\ell \boldsymbol e_\ell \right) \times \left(b_i c_j \varepsilon_{ijk} \boldsymbol e_k \right) \\
    & = a_\ell b_i c_j \varepsilon_{ijk} \varepsilon_{\ell km} \boldsymbol e_m \\
    & = a_\ell b_i c_j \varepsilon_{ijk} \varepsilon_{m \ell k} \boldsymbol e_m \\
    & = a_\ell b_i c_j \left( \delta_{im}\delta_{j \ell} - \delta_{i \ell}\delta_{jm} \right) \boldsymbol e_m \\
    & = \underset{\text{Let } i = m,\ j = \ell}{\underline{a_j b_i c_j \boldsymbol e_i}} - \underset{\text{Let } i = \ell,\ j = m}{\underline{a_i b_i c_j \boldsymbol e_j}} \\
    & = \boldsymbol b  \left(\boldsymbol a \cdot \boldsymbol c \right) - \boldsymbol c \left(\boldsymbol a \cdot \boldsymbol b \right).
\end{align*}

\emph{助记:``back cab'' (后面的出租车)}

\subsection*{(2) Two repeated indices
2个指标重复}\label{two-repeated-indices-2ux4e2aux6307ux6807ux91cdux590d}

Let \(i=\ell, \ j = m\), and
\[\varepsilon_{ijk}\varepsilon^{ijn} = {\delta_j}^j {\delta_k}^n - {\delta_j}^n {\delta_k}^j = 3 {\delta_k}^n - {\delta_k}^n = 2 {\delta_k}^n.\]

\subsection*{(3) Three repeated indices
3个指标重复}\label{three-repeated-indices-3ux4e2aux6307ux6807ux91cdux590d}

Let \(i=\ell, \ j = m, \ k = n\), and
\[\varepsilon_{ijk}\varepsilon^{ijk} = 2 {\delta_k}^k = 6 = 3!.\]

Under \(n\) dimensions, where \(i_1, i_2, \cdots, i_n\) takes the values
\(1, 2, \cdots, n\), there is
\[\varepsilon_{i_1 i_2 \cdots i_n}\varepsilon^{i_1 i_2 \cdots i_n} = n!,\]
where the exclamation mark (\(!\)) denotes the factorial (阶乘).

\subsection*{(4) An example}\label{an-example}

A rigid body (刚体) is rotating at a constant angular velocity
\(\boldsymbol \omega_0\), and the velocity is
\[\boldsymbol v = \boldsymbol \omega_0 \times \boldsymbol r.\]

\begin{quote}
作业:用Levi-Civita符号展开上式。
\end{quote}

The curl of \(\boldsymbol v\) is \begin{align*}
    \nabla \times \left( \boldsymbol \omega_0 \times \boldsymbol r\right)
    & = {\partial \over \partial x_\ell} \boldsymbol e_\ell \times \left(\omega_{0, i} x_j \varepsilon_{ijk} \boldsymbol e_k \right) \\
    & = \omega_{0, i} {\partial x_j \over \partial x_\ell} \varepsilon_{ijk} \varepsilon_{lkm} \boldsymbol e_m \\
    & = \omega_{0, i} \delta_{j \ell} \varepsilon_{ijk} \varepsilon_{lkm} \boldsymbol e_m \\
    & = \omega_{0, i} \varepsilon_{ijk} \varepsilon_{jkm} \boldsymbol e_m \\
    & = \omega_{0, i} \varepsilon_{ijk} \varepsilon_{mjk} \boldsymbol e_m \\
    & = \omega_{0, i} \cdot 2 \delta_{im} \cdot \boldsymbol e_m \\
    & = 2 \omega_{0, i} \boldsymbol e_i \\
    & = 2 \boldsymbol \omega_0.
\end{align*}

\emph{This is ``Error in Beauty'' -- Sommerfeld}

\begin{quote}
作业:Feynman 的口诀:\begin{align*}
\nabla \times \left( \boldsymbol a \times \boldsymbol b \right) & = \nabla_{\boldsymbol a} \times \left( \boldsymbol a \times \boldsymbol b \right) + \nabla_{\boldsymbol b} \times \left( \boldsymbol a \times \boldsymbol b \right) \\
& = \boldsymbol a \left(\nabla_{\boldsymbol a} \cdot \boldsymbol b \right) - \boldsymbol b \left(\nabla_{\boldsymbol a} \cdot \boldsymbol a \right) + \boldsymbol a \left(\nabla_{\boldsymbol b} \cdot \boldsymbol b \right) - \boldsymbol b \left(\nabla_{\boldsymbol b} \cdot \boldsymbol a \right).
\end{align*}

表示上式。
\end{quote}
