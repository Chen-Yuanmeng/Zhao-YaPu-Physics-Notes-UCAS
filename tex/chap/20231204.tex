\chapter{2023/12/04}\label{20231204}

\section{Deduction of the operators of momentum, energy and angular
momentum
动量、能量、角动量算符的推导}\label{deduction-of-the-operators-of-momentum-energy-and-angular-momentum-ux52a8ux91cfux80fdux91cfux89d2ux52a8ux91cfux7b97ux7b26ux7684ux63a8ux5bfc}

\begin{itemize}
\tightlist{}
\item
  The wave function 波函数
  \[\psi = \mathrm e^{\mathrm i(\boldsymbol k \cdot \boldsymbol r - \omega t)} = \mathrm e^{\mathrm {i \over \hbar}(\boldsymbol p \cdot \boldsymbol r - E t)}\]

  Because the wave function is an indication of probability, it must
  exists that \[\iiint _{(V)} |\psi|^2 \mathrm d\tau=1.\]
\item
  The derivation of \(\hat p\) 推导动量算符 \(\hat p\)

  \[\nabla \psi = {\partial \psi \over \partial \boldsymbol r} = {\mathrm i \over \hbar}\boldsymbol p \psi\] \[\Rightarrow \boldsymbol p \psi = -\mathrm i \hbar \nabla \psi\] \[\Rightarrow \hat p = -\mathrm i \hbar \nabla\]

  From this we can see that momentum is linked with space
  (动量与空间相联系), and consequently, the conservation of momentum
  (动量守恒) is related to spatial translation invariance
  (空间平移不变性).
\item
  The derivation of \(\hat E\) 推导能量算符 \(\hat E\)

  \[{\partial \psi \over \partial t} = -{\mathrm i \over \hbar}E \psi\] \[\Rightarrow E \psi = \mathrm i \hbar {\partial \psi \over \partial t}\] \[\Rightarrow \hat E = \mathrm i \hbar {\partial \over \partial t}\]

  From this we can see that energy is linked with time
  (能量与时间相联系), and consequently, the conservation of energy
  (能量守恒) is related to time translation invariance (时间平移不变性).
\item
  The derivation of \(\hat{\boldsymbol{L}}\)
  推导角动量算符\(\hat{\boldsymbol{L}}\)

  From \(\hat p = -\mathrm i \hbar \nabla\), we know that
  \[\hat{\boldsymbol{L}} = \hat{\boldsymbol{r}} \times \hat{\boldsymbol{p}} = \hat{\boldsymbol{r}} \times \left( -\mathrm i \hbar \nabla \right) = -\mathrm i \hbar \hat{\boldsymbol{r}} \times \nabla.\]

  From this we can see that angular momentum is linked with space
  (角动量与时空相联系), and consequently, the conservation of angular
  momentum (角动量守恒) is related to spacial rotation invariance
  (时间平移不变性).
\end{itemize}

\section{Noether's theorem (1918)
诺特定理}\label{noethers-theorem-1918-ux8bfaux7279ux5b9aux7406}

by Emmy Noether (1882\textasciitilde1935)

\subsection*{(0) Introduction 引入}\label{introduction-ux5f15ux5165}

\emph{Freeman Dyson: Birds and Frogs (2009)}

\begin{itemize}
\tightlist{}
\item
  There are two kinds of scientists:

  \begin{itemize}
\tightlist{}
  \item
    \textbf{Birds}: they fly high, see the landscape and have delight in
    unifying
  \item
    \textbf{Frogs}: they reside in the mud and see the details
  \end{itemize}
\item
  Four jokes by nature:

  \begin{itemize}
\tightlist{}
  \item
    the square root of minus one that the physicist Erwin Schrödinger
    put into his wave equation when he invented wave mechanics
    被放入薛定谔波函数的-1的平方根
  \item
    the precise linearity of quantum mechanics, the fact that the
    possible states of any physical object form a linear space
    量子力学的精确线性,即物理对象的所有可能状态构成线性空间
  \item
    the existence of quasi-crystals 拟晶体的存在
  \item
    a similarity in behavior between quasi-crystals and the zeros of the
    Riemann Zeta function 拟晶体和黎曼 \(\zeta\) 函数在行为上的相似性
  \end{itemize}
\end{itemize}

\subsection*{(1) Contents 内容}\label{contents-ux5185ux5bb9-1}

\textbf{Every differentiable symmetry of the action of a physical system
with conservative forces has a corresponding conservation law.}

每个连续对称性都有着相应的守恒定律。

\subsection*{(2) Examples of Noether's theorem
诺特定理的例证}\label{examples-of-noethers-theorem-ux8bfaux7279ux5b9aux7406ux7684ux4f8bux8bc1}

We will focus on five pairs of conserved physical quantities and
symmetry:

\begin{center}
    \begin{tabular}{|c|c|}
        \hline
        \textbf{Physical quantities (物理量)} & \textbf{Corresponding symmetry (对应的对称性)} \\
        \hline
        Momentum (动量) & Spatial translation invariance (空间平移不变性) \\
        \hline
        Angular momentum (角动量) & Spatial rotation invariance (空间旋转不变性) \\
        \hline
        Energy (能量) & Time translation invariance (时间平移不变性) \\
        \hline
        Charge (电荷) & Gauge invariance (规范不变性) \\
        \hline
        Mass-Energy (质-能守恒) & Time translation invariance (时间平移不变性) \\
        \hline
    \end{tabular}
    \captionof{table}{Correspondence between Physical Quantities and Symmetry}
\end{center}

\paragraph{i. Momentum (动量) \(\Leftrightarrow\) Spatial
translation invariance
(空间平移不变性)}\label{momentum-ux52a8ux91cf-leftrightarrow-spatial-translation-invariance-ux7a7aux95f4ux5e73ux79fbux4e0dux53d8ux6027}

In a closed system, suppose we have a parallel infinitesimal
displacement (平行无穷小位移)
\(\boldsymbol{\varepsilon} = \delta q_\alpha\) (constant), and then we
know
\[\delta L = \sum_{\alpha} \frac{\partial L}{\partial q_\alpha} \delta \boldsymbol{q}_\alpha = \boldsymbol{\varepsilon} \cdot \sum_{\alpha} \frac{\mathrm{d}}{\mathrm{d}t} \frac{\partial L}{\partial q_\alpha} \delta \dot{\boldsymbol{q}}_\alpha = \boldsymbol{\varepsilon} \cdot \frac{\mathrm{d}}{\mathrm{d}t} \sum_{\alpha} p_\alpha = \boldsymbol{\varepsilon} \cdot \frac{\mathrm{d} \boldsymbol{p}}{\mathrm{d}t} = \boldsymbol{0}.\]

Because \(\boldsymbol{\varepsilon} \neq \boldsymbol{0}\), we have
\[\frac{\mathrm{d} \boldsymbol{p}}{\mathrm{d}t} = \boldsymbol{0},\]which
means that \(\boldsymbol{p}\) is conserved.

Through this, we have deducted momentum conservation from spatial
translation invariance.

\paragraph{ii. Angular Momentum (角动量)
\(\Leftrightarrow\) Spatial rotation invariance (空间旋转不变性) /
isotropy
(各向同性)}\label{angular-momentum-ux89d2ux52a8ux91cf-leftrightarrow-spatial-rotation-invariance-ux7a7aux95f4ux65cbux8f6cux4e0dux53d8ux6027-isotropy-ux5404ux5411ux540cux6027}

Suppose we have a infinitesimal rotation angle (无穷小旋转角)
\(\delta \boldsymbol{\varphi}\) (constant), and then we know
\[\delta \boldsymbol{r}_\alpha = \delta \boldsymbol{\varphi} \times \boldsymbol{r}_\alpha,\]
and
\[\delta \boldsymbol{v}_\alpha = \delta \dot{\boldsymbol{r}}_\alpha = \delta \boldsymbol{\varphi} \times \dot{\boldsymbol{r}}_\alpha.\]

\begin{align*}
    \delta L & = \sum_{\alpha} \left( \frac{\partial L}{\partial \boldsymbol{r}_\alpha} \cdot \delta \boldsymbol{r}_\alpha + \frac{\partial L}{\partial \dot{\boldsymbol{r}}_\alpha} \cdot \delta \dot{\boldsymbol{r}}_\alpha \right) \\
    & = \sum_{\alpha} \dot{\boldsymbol{p}}_\alpha \cdot \left( \delta \boldsymbol{\varphi} \times \boldsymbol{r}_\alpha \right) + \sum_{\alpha} \boldsymbol{p}_\alpha \cdot \left( \delta \boldsymbol{\varphi} \times \dot{\boldsymbol{r}}_\alpha \right) & \quad (1)\\
    & = \delta \boldsymbol{\varphi} \cdot \left[ \sum_{\alpha} \left( \dot{\boldsymbol{r}}_\alpha \times \boldsymbol{p}_\alpha + \boldsymbol{r}_\alpha \times \dot{\boldsymbol{p}}_\alpha \right) \right] & \quad (2)\\
    & = \delta \boldsymbol{\varphi} \cdot \left[ \sum_{\alpha} \frac{\mathrm{d}}{\mathrm{d}t} (\boldsymbol{r}_\alpha \times \boldsymbol{p}_\alpha) \right] \\
    & = \delta \boldsymbol{\varphi} \cdot \frac{\mathrm{d} \boldsymbol{M}}{\mathrm{d}t} \\
    & = \boldsymbol{0}.
\end{align*}

Note: From \((1)\) to \((2)\), we used the equation
\(\boldsymbol a \cdot (\boldsymbol b \times \boldsymbol c) = \boldsymbol c \cdot (\boldsymbol a \times \boldsymbol b) = \boldsymbol b \cdot (\boldsymbol c \times \boldsymbol a).\)

Because \(\delta \boldsymbol{\varphi} \neq \boldsymbol{0}\), we have
\[\frac{\mathrm{d} \boldsymbol{M}}{\mathrm{d}t} = 0,\] which means that
\(\displaystyle \boldsymbol{M} = \sum_{\alpha} \boldsymbol{r}_\alpha \times \boldsymbol{p}_\alpha\)
is conserved. This indicates spatial rotation invariance
(空间旋转不变性).

\paragraph{iii. Energy (能量) \(\Leftrightarrow\) Time
translation invariance
(时间平移不变性)}\label{energy-ux80fdux91cf-leftrightarrow-time-translation-invariance-ux65f6ux95f4ux5e73ux79fbux4e0dux53d8ux6027}

Suppose there is a time-translation-invariant system, where
\(L(q, \dot{q}, t) = L(q, \dot{q})\) doesn't depend on time \(t\)
explicitly. Then we have \begin{align*}
    \frac{\mathrm{d} L}{\mathrm{d}t} & = \sum_{\alpha} \left( \frac{\partial L}{\partial q_\alpha} \dot{q}_\alpha + \frac{\partial L}{\partial \dot q_\alpha} \ddot{q}_\alpha \right)  = \sum_{\alpha} \left( \frac{\mathrm{d}}{\mathrm{d}t} \frac{\partial L}{\partial \dot q_\alpha} \dot{q}_\alpha + \frac{\partial L}{\partial \dot q_\alpha} \ddot{q_\alpha} \right) = \frac{\mathrm{d}}{\mathrm{d}t} \sum_{\alpha} \left( \frac{\partial L}{\partial \dot q_\alpha} \dot{q}_\alpha \right) \\
    & = \frac{\mathrm{d}}{\mathrm{d}t} \sum_{\alpha} \left( \frac{\partial T}{\partial \dot q_\alpha} \dot{q}_\alpha \right) = \frac{\mathrm{d}}{\mathrm{d}t} \sum_{\alpha} \left( m \dot{q}_\alpha \dot{q}_\alpha \right) = \frac{\mathrm{d}}{\mathrm{d}t} (2T).
\end{align*}

From this, we know that
\[\frac{\mathrm{d} L}{\mathrm{d}t} = \frac{\mathrm{d} (2T)}{\mathrm{d}t},\]
and
\[\frac{\mathrm{d} (2T - L)}{\mathrm{d}t} = \frac{\mathrm{d} [2T - (T - V)]}{\mathrm{d}t} = \frac{\mathrm{d} (T + V)}{\mathrm{d}t} = \frac{\mathrm{d} E}{\mathrm{d}t} = 0.\]

That is to say, energy \(E\) is conserved, and we have the energy
conservation law.

\paragraph{iv. Charge (电荷) \(\Leftrightarrow\) Gauge
symmetry
(规范对称性)}\label{charge-ux7535ux8377-leftrightarrow-gauge-symmetry-ux89c4ux8303ux5bf9ux79f0ux6027}

\emph{Fusion: Education + Research}

In 2022, \emph{Physical Review Letters} (PRL) published \emph{Machine Learning Hidden Symmetries} (\url{https://doi.org/10.1103/PhysRevLett.128.180201}).

In \emph{More is different} (\url{https://doi.org/10.1126/science.177.4047.393}) by P. W. Anderson (1923-2020)*: ``It is only slightly overstating the case to say that physics is the study of symmetry.''

\emph{译者注:赵爹上课没写only这个词,于是整句话意思完全不一样了(笑哭)}

\begin{itemize}
\tightlist{}
\item
  Derivation of charge conservation 电荷守恒的导出

  From the Maxwell's equations we know that
  \[\nabla \cdot \boldsymbol E = \dfrac{\rho}{\varepsilon_0}\] and
  \[\nabla \times \boldsymbol{B} = \mu_0 \left( \boldsymbol{j} + \varepsilon_0 \frac{\partial \boldsymbol{E}}{\partial t} \right).\]

  The divergence of a curl is zero (旋度无散), and we have
  \begin{align*}
        \nabla \cdot \left( \nabla \times \boldsymbol{B} \right) & = \mu_0 \nabla \cdot \boldsymbol{j} + \mu_0 \varepsilon_0 \nabla \cdot \frac{\partial \boldsymbol{E}}{\partial t} \\
        & = \mu_0 \left[ \nabla \cdot \boldsymbol{j} + \varepsilon_0 \frac{\partial \left( \nabla \cdot \boldsymbol{E} \right)}{\partial t} \right] \\
        & = \mu_0 \left[ \nabla \cdot \boldsymbol{j} + \varepsilon_0 \frac{\partial}{\partial t} \left( \frac{\rho}{\varepsilon_0} \right) \right] \\
        & = \mu_0 \left[ \nabla \cdot \boldsymbol{j} + \frac{\partial \rho}{\partial t} \right] \\
        & = 0.
    \end{align*}

  Thus we have
  \[\nabla \cdot \boldsymbol{j} + \frac{\partial \rho}{\partial t} = 0,\]
  which is called \textbf{the charge density continuity equation
  (电荷连续性方程).}

  Written in covariant and contravariant (协变和逆变) form is
  \[\partial_{\mu} J^{\mu} = 0.\]

  Here,
  \[\partial_{\mu} = \left( \frac{1}{c} \frac{\partial}{\partial t}, \frac{\partial}{\partial x}, \frac{\partial}{\partial y}, \frac{\partial}{\partial z} \right),\]
  and \[J^{\mu} = \left( c \rho, v_x \rho, v_y \rho, v_z \rho \right).\]
\item
  Gauge invariance 规范不变性

  For this part, please refer to the notes on 2023/10/16.
\item
  One-dimensional manifold 一维流形 \(\operatorname U(1)\)

  Here \(\operatorname U(1)\) stands for \textbf{unitarity 酉(或幺正)}.

  \[\operatorname U(\theta) = \mathrm{e}^{\mathrm{i} \theta} = \cos \theta + \mathrm{i} \sin \theta \quad (*)\]

  The conjugate (共轭) of \((*)\) is
  \[\operatorname U^*(\theta) = \mathrm{e}^{- \mathrm{i} \theta} = \cos \theta - \mathrm{i} \sin \theta.\]

  Note that this is a multiply-connected region (多连通域).

  \emph{译者注:这里赵老师的板书是multiply region,经查似乎有误。}
\end{itemize}

\paragraph{v. Mass-Energy (质-能守恒) \(\Leftrightarrow\)
Time translation invariance
(时间平移不变性)}\label{mass-energy-ux8d28-ux80fdux5b88ux6052-leftrightarrow-time-translation-invariance-ux65f6ux95f4ux5e73ux79fbux4e0dux53d8ux6027}

According to relativity theory,
\[E = \gamma mc^2 = \frac{mc^2}{\sqrt{1 - v^2 / c^2}}.\]

Expand the Taylor series of the above equation (对上式做泰勒展开), and
we get
\[E = mc^2 \left( 1 + \frac{1}{2} \frac{v^2}{c^2} \right) = mc^2 + \frac{1}{2} mv^2.\]

This is a hidden math conservation.

Mass conservation has a similar form to charge conservation, but here
\(\boldsymbol{J}\) is the mass flow:
\[\nabla \cdot \boldsymbol{J} + \frac{\partial \rho}{\partial t} = 0,\]
and it can also be written in covariant and contravariant form:
\[\partial_{\mu} J^{\mu} = 0.\]
