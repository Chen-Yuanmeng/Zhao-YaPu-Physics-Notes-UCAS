\chapter{2023/11/27}\label{20231127}

\section{Metric tensors
度规张量}\label{metric-tensors-ux5ea6ux89c4ux5f20ux91cf-1}

\emph{Feynman去坐出租车,他不知道应该去哪个校区,于是他跟司机说:``去那个昨天有很多人去的地方,他们嘴上都说着什么
\(\mu\nu\) 之类的东西。''}

\subsection*{(1) Definition 定义}\label{definition-ux5b9aux4e49-1}

Define
\[\mathbf{g} = g_{\mu\nu} \boldsymbol{e}^{\mu} \otimes \boldsymbol{e}^{\nu} = \frac{\partial \boldsymbol{r}}{\partial x_\mu} \cdot \frac{\partial \boldsymbol{r}}{\partial x_\nu} \boldsymbol{e}^{\mu} \otimes \boldsymbol{e}^{\nu}.\]

\subsection*{(2) Examples 例子}\label{examples-ux4f8bux5b50}

\begin{itemize}
\tightlist{}
\item
  Lorentzian metrics from relativity 相对论的洛伦兹度量

  In this case, the space-time interval can be written as
  \[\mathrm{d} s^2 = c^2 \mathrm{d}t^2 - \mathrm{d} x^2 - \mathrm{d} y^2 - \mathrm{d} z^2 = g_{\mu\nu} \mathrm{d}x^{\mu} \mathrm{d}x^{\nu},\]
  where \(\mathrm{d} s^2\) stands for \((\mathrm{d} s)^2\).

  From this we can see \[\mathbf{g} = \begin{pmatrix}
        1 & 0 & 0 & 0 \\
        0 & -1 & 0 & 0 \\
        0 & 0 & -1 & 0 \\
        0 & 0 & 0 & -1
    \end{pmatrix}.\]

  Also from this we can have
  \begin{align*} \mathrm{d} s^2 & = c^2 \mathrm{d}t^2 \left[ 1 - \frac{1}{c^2} \left( \frac{\mathrm{d}x}{\mathrm{d}t} \right)^2 - \frac{1}{c^2} \left( \frac{\mathrm{d}y}{\mathrm{d}t} \right)^2 - \frac{1}{c^2} \left( \frac{\mathrm{d}z}{\mathrm{d}t} \right)^2 \right] \\
        & = c^2 \mathrm{d}t^2 \left( 1 - {v^2 \over c^2} \right) \\
        & = c^2 \mathrm{d} \tau^2,
    \end{align*} where \(\tau\) is the \textbf{proper time (固有时)},
  and
  \[\frac{\mathrm{d}t}{\mathrm{d} \tau} = \gamma = \frac{1}{\sqrt{1 - \dfrac{v^2}{c^2}}}.\]
\item
  Cartesian coordinates 笛卡尔坐标系

  In this situation we have \(\boldsymbol{r} = (x, y, z)\), and thus we
  have \[\frac{\partial \boldsymbol{r}}{\partial x} = (1, 0, 0),\]
  \[\frac{\partial \boldsymbol{r}}{\partial y} = (0, 1, 0),\]
  \[\frac{\partial \boldsymbol{r}}{\partial z} = (0, 0, 1).\]

  From this we know \[g_{xx} = g_{yy} = g_{zz} = 1, \] and
  \[g_{xy} = g_{yx} = g_{xz} = g_{zx} = g_{yz} = g_{yz} =0.\]

  \[\mathbf{g} = \begin{pmatrix}
        1 & 0 & 0 \\
        0 & 1 & 0 \\
        0 & 0 & 1
    \end{pmatrix}.\]
\item
  Cylindrical coordinates 柱坐标

  In this situation we have
  \(\boldsymbol{r} = (r \cos \theta, r \sin \theta, z)\), or
  \[\mathrm{d} s^2 = \mathrm{d}r^2 + r^2 \mathrm{d} \theta^2 + \mathrm{d} z^2 = g_{\mu\nu} \mathrm{d}x^{\mu} \mathrm{d}x^{\nu}.\]

  Thus, we have
  \[\frac{\partial \boldsymbol{r}}{\partial r} = (\cos \theta, \sin \theta, 0),\]
  \[\frac{\partial \boldsymbol{r}}{\partial \theta} = (-r \sin \theta, r \cos \theta, 0),\]
  \[\frac{\partial \boldsymbol{r}}{\partial z} = (0, 0, 1).\]

  From this we know \[g_{rr} = \cos^2 \theta + \sin^2 \theta = 1,\]
  \[g_{\theta \theta} = (-r \sin \theta)^2 + (r \cos \theta)^2 = r^2,\]
  \[g_{zz} = 1.\]

  \[\mathbf{g} = \begin{pmatrix}
        1 & 0 & 0 \\
        0 & r^2 & 0 \\
        0 & 0 & 1
    \end{pmatrix}.\]
\end{itemize}

\section[Tensor form of Lagrangian kinetic energy 拉格朗日动能的张量形式]{Tensor form of kinetic energy in Lagrangian systems 拉格朗日系统中动能的张量形式}\label{tensor-form-of-kinetic-energy-in-lagrangian-systems-ux62c9ux683cux6717ux65e5ux7cfbux7edfux4e2dux52a8ux80fdux7684ux5f20ux91cfux5f62ux5f0f}

Suppose we have \(N\) particles in the system, and the kinetic energy is
\[T = \sum_{i = 1}^{N} {1 \over 2} m_i \boldsymbol{v}_i \cdot \boldsymbol{v}_i.\]

Usually we have the position of a particle
\(\boldsymbol{r}_i = \boldsymbol{r}_i(q_1, q_2, \dots, q_n; t)\). Take
the total derivative (全微分) of \(\boldsymbol{r}_i\) and we have
\[\boldsymbol{v}_i = {\mathrm{d} \boldsymbol{r}_i \over \mathrm{d}t} = {\partial \boldsymbol{r}_i \over \partial t} + \sum_{\alpha = 1}^{n} {\partial \boldsymbol{r}_i \over \partial q_\alpha} {\mathrm{d} q_\alpha \over \mathrm{d}t}.\]

When the positions \(\boldsymbol{r}_i\) doesn't explicitly contain time
\(t\), we have \[{\partial \boldsymbol{r}_i \over \partial t} = 0\] and
\[\boldsymbol{v}_i =  \sum_{\alpha = 1}^{n} {\partial \boldsymbol{r}_i \over \partial q_\alpha} {\mathrm{d} q_\alpha \over \mathrm{d}t}.\]

Thus \begin{align*}
    T & = \sum_{i = 1}^{N} {1 \over 2} m_i \left( \sum_{\alpha = 1}^{n} {\partial \boldsymbol{r}_i \over \partial q_\alpha} {\mathrm{d} q_\alpha \over \mathrm{d}t} \right) \cdot \left( \sum_{\beta = 1}^{n} {\partial \boldsymbol{r}_i \over \partial q_\beta} {\mathrm{d} q_\beta \over \mathrm{d}t} \right) \\
    & = {1 \over 2} \sum_{\alpha, \beta = 1}^{n} \sum_{i = 1}^{N} m_i {\partial \boldsymbol{r}_i \over \partial q_\alpha} {\partial \boldsymbol{r}_i \over \partial q_\beta} {\mathrm{d} q_\alpha \over \mathrm{d}t} {\mathrm{d} q_\beta \over \mathrm{d}t} \\
    & = {1 \over 2} \sum_{\alpha, \beta = 1}^{n} a_{\alpha\beta} \dot{q}_\alpha \dot{q}_\beta \\
    & = {1 \over 2} \sum_{\alpha, \beta = 1}^{n} \dot{\boldsymbol{q}}_\alpha^{T} \cdot a_{\alpha \beta} \cdot \dot{\boldsymbol{q}}_\beta.
\end{align*}

\section{Rayleigh dissipation function
瑞利耗散函数}\label{rayleigh-dissipation-function-ux745eux5229ux8017ux6563ux51fdux6570}

Suppose we have \(f = - \mu \dot{x}\), and we define the Rayleigh
dissipation function
\[R = {1 \over 2} \mu \dot{x}^2 = {1 \over 2} |f| \dot{x}.\]

When \(R\) is added to the Euler-Lagrange equation, we acquire
\[{\mathrm{d} \over \mathrm{d}t} {\partial L \over \partial \dot q_\alpha} - {\partial L \over \partial q_\alpha} = - {\partial R \over \partial \dot q_\alpha}.\]

Also, in tensor form, we can have
\[R = {1 \over 2} \boldsymbol{q}^T \cdot \mathbf{\mu} \cdot \boldsymbol{q}.\]

The dimension of \(R\) is the same as that of power:
\[[R] = [F] \cdot [v] = [P].\]
