\chapter{2023/9/13}\label{20230913}

\section{Something about Hamilton Canonical Equations 关于哈密顿正则方程的一点讨论}\label{something-about-hamilton-canonical-equations-ux5173ux4e8eux54c8ux5bc6ux987fux6b63ux5219ux65b9ux7a0bux7684ux4e00ux70b9ux8ba8ux8bba}

The second equation can be derived to this form \[\dot p_\alpha=-\dfrac{\partial H}{\partial q_\alpha}=-\dfrac{\partial (T+V)}{\partial q_\alpha}=-\dfrac{\partial V}{\partial q_\alpha},\] in which \(T=\dfrac{p_\alpha ^2} {2m}\) is independent of \(q_\alpha\).

This is in accordance with \(F(x) = -\dfrac{\mathrm dV(x) }{\mathrm dx}\) in Newtonian mechanics.

\section{EPR paradox EPR详谬}\label{epr-paradox-eprux8be6ux8c2c}

\begin{quote}
作业1:什么是EPR paradox?
\end{quote}

Albert Einstein VS N. Bohr: The Great Debate 玻尔和爱因斯坦论战

\section{Something about quantum mechanics 关于量子力学的一点讨论}\label{something-about-quantum-mechanics-ux5173ux4e8eux91cfux5b50ux529bux5b66ux7684ux4e00ux70b9ux8ba8ux8bba}

\begin{itemize}
\tightlist{}
    
\item
    In 1905, Einstein, in his famous \emph{On the Electrodynamics of Moving Bodies} (《论动体的电动力学》) essay, pointed out two famous postulates (公设): length compression (尺缩现象) and time dilation (钟慢现象).
\item
    Heisenberg's Uncertainty Principle: 不确定性关系/测不准原理

    Heisenberg's Uncertainty Principle states that \[\Delta x \cdot \Delta p_x \geq {\hbar \over 2},\] \[\Delta E \cdot \Delta t \geq {\hbar \over 2},\] \[\Delta \tau \cdot \Delta \theta \geq {\hbar \over 2}.\]
\item
  Occam's Razor: 奥卡姆剃刀原理

  Entities should not be multiplied unnecessarily. 如无必要,勿增实体。
\end{itemize}

\section{Newtonianism VS Darwinism 牛顿主义和达尔文主义}\label{newtonianism-vs-darwinism-ux725bux987fux4e3bux4e49ux548cux8fbeux5c14ux6587ux4e3bux4e49}

\begin{itemize}
\tightlist{}
\item
  Newtonianism 牛顿主义: linear (线性的), simple (简洁的), superposable (空间可叠加的)
\item
  Darwinism 达尔文主义: non-linear (非线性的), complex (复杂的), non-superposable (空间不可叠加的)
\end{itemize}

\begin{quote}
作业2:康德的``二律背反''是什么?
\end{quote}

\begin{quote}
作业3:了解新冠肺炎和社会达尔文主义
\end{quote}

\section{About the Hamiltonian operator (哈密顿算子) \(\nabla\)}\label{about-the-hamiltonian-operator-ux54c8ux5bc6ux987fux7b97ux5b50-nabla}

\emph{Hamilton 23岁就当上了正教授!要学习他!}

The notation \(\nabla\) is pronounced as \texttt{del} or \texttt{nabla}.

We define \(\nabla\) under three-dimensional Cartesian coordinate (三维笛卡尔坐标系), the base/unit vector (基矢/单位矢量) of three orthogonal (正交的) directions being \(\hat i\), \(\hat j\) and \(\hat k\) (also sometimes written as \(\boldsymbol e_x\), \(\boldsymbol e_y\) and \(\boldsymbol e_z\)): \[\nabla= {\partial \over \partial x}\hat i + {\partial \over \partial y}\hat j + {\partial \over \partial z}\hat k.\]

Do note that this Hamiltonian operator is not the same as the one by the same name in quantum mechanics (与量子力学中的哈密顿算符做区分), which, in quantum mechanics, is \(\displaystyle H= -{\hbar^2 \over 2m} \nabla^2 + U.\)

Then we can define:

\subsection*{(1) Gradient 梯度}\label{gradient-ux68afux5ea6}

\begin{itemize}
\tightlist{}
\item
  The gradient of a scalar becomes a vector (essentially a first-order tensor) 对标量求梯度得到矢量(其实质为一阶张量):  \[\mathrm{grad} \ \varphi = \nabla \varphi = {\partial \varphi \over \partial x}\hat i + {\partial \varphi \over \partial y}\hat j + {\partial \varphi \over \partial z}\hat k.\]
  For example: \(\boldsymbol F = - \nabla U\).
\item
    The gradient of a vector becomes a second-order tensor 对矢量求梯度得到二阶张量:
    \[\mathrm{grad} \boldsymbol A = \nabla \boldsymbol A = ({\partial \over \partial x_i}\hat i) \otimes (A_j \hat j) = {\partial A_j \over \partial x_i} (\hat i \otimes \hat j) = \begin{bmatrix}
        \dfrac{\partial A_x}{\partial x} & \dfrac{\partial A_y}{\partial x} & \dfrac{\partial A_z}{\partial x} \\[1.5ex]
        \dfrac{\partial A_x}{\partial y} & \dfrac{\partial A_y}{\partial y} & \dfrac{\partial A_z}{\partial y} \\[1.5ex]
        \dfrac{\partial A_x}{\partial z} & \dfrac{\partial A_y}{\partial z} & \dfrac{\partial A_z}{\partial z} \\[1.5ex]
    \end{bmatrix}.\]
\end{itemize}

Note: The equation above uses the Einstein summation convention (爱因斯坦求和约定). For more reference about it, see the notes on 2023/10/16.

\subsection*{(2) Divergence 散度}\label{divergence-ux6563ux5ea6}

\[\mathrm{div} \boldsymbol A = \nabla \cdot \boldsymbol A = {\partial A_x \over \partial x} + {\partial A_y \over \partial y} + {\partial A_z \over \partial z}.\]

In its essence we are performing the act of the deduction of \(\boldsymbol A\) (\(\boldsymbol A\)的降阶) when we calculate
\(\mathrm{div} \boldsymbol A\).

Note that we can only calculate the divergence of \textbf{vectors}, not \textbf{scalars}.

\subsection*{(3) Curl 旋度}\label{curl-ux65cbux5ea6}

\[\mathrm{curl} \boldsymbol A = \nabla \times \boldsymbol A = \begin{vmatrix}
    \hat i & \hat j & \hat k \\
    \dfrac{\partial}{\partial x} & \dfrac{\partial}{\partial y} & \dfrac{\partial}{\partial z} \\
    A_x & A_y & A_z \\
\end{vmatrix}.\]

\begin{quote}
作业4:证明标量的梯度无旋、矢量的旋度无散
\end{quote}

\begin{quote}
作业5:把Maxwell方程组写成分量形式
\end{quote}

\subsection*{(4) Some practice 应用}\label{some-practice-ux5e94ux7528}

In \textbf{field theory} (场论), we define \(\boldsymbol r = x \hat i + y \hat j + z \hat k\).

\paragraph{i. \(\boldsymbol r\) 的散度}\label{boldsymbol-r-ux7684ux6563ux5ea6}

\[\nabla \cdot \boldsymbol r = ({\partial \over \partial x}\hat i + {\partial \over \partial y}\hat j + {\partial \over \partial z}\hat k) \cdot (x \hat i + y \hat j + z \hat k) = {\partial x \over \partial x} + {\partial y \over \partial y} + {\partial z \over \partial z} =3.\]

\paragraph{ii. The gravitational field 引力场}\label{the-gravitational-field-ux5f15ux529bux573a}

\[\nabla r = \dfrac{\boldsymbol r}{|\boldsymbol r|} = \hat{\boldsymbol r}.\]

In the gravitational field, we have \[U(r) = -{Gm_1m_2 \over r}.\]

Thus we have \[\boldsymbol F = -\nabla U = Gm_1m_2 \cdot \nabla(\dfrac{1}{r}), \] in which \[\nabla(\dfrac{1}{r})={-\nabla r \over r^2}, \] and consequently \[\boldsymbol F = - {G m_1 m_2 \over r^3}\boldsymbol r = - {G m_1 m_2 \over r^2} \hat {\boldsymbol r}.\]

\paragraph{iii. \(\boldsymbol r\) 的梯度}\label{boldsymbol-r-ux7684ux68afux5ea6}

\begin{align*}
    \nabla \boldsymbol r & = \nabla \otimes \boldsymbol r \\
    & = ({\partial \over \partial x}\hat i + {\partial \over \partial y}\hat j + {\partial \over \partial z}\hat k) \otimes (x \hat i + y \hat j + z \hat k) \\
    & = {\partial x \over \partial x} (\hat i \otimes \hat i) + {\partial y \over \partial y}(\hat j \otimes \hat j) + {\partial z \over \partial z}(\hat k \otimes \hat k)\\
    & = (\hat i \otimes \hat i) + (\hat j \otimes \hat j) + (\hat k \otimes \hat k) \\
    & =\mathbf I \\
    & = \begin{pmatrix}
        1 & 0 & 0 \\
        0 & 1 & 0 \\
        0 & 0 & 1 \\
    \end{pmatrix}.
\end{align*}

\paragraph{iv. The Laplacian operator 拉普拉斯算符}\label{the-laplacian-operator-ux62c9ux666eux62c9ux65afux7b97ux7b26}

\par

The Laplacian operator is linear (线性的). It is defined as \[\nabla^2 = \Delta = \nabla \cdot \nabla = {\partial^2 \over \partial x^2} + {\partial^2 \over \partial y^2} + {\partial^2 \over \partial z^2}.\]

For example, \[\nabla^2 r = \nabla \cdot \nabla r = \nabla \cdot (\nabla r) = {2 \over r}\]

\begin{quote}
作业6:推导上式
\end{quote}

\section{Discussion about some equations 关于公式的一些讨论}\label{discussion-about-some-equations-ux5173ux4e8eux516cux5f0fux7684ux4e00ux4e9bux8ba8ux8bba}

\subsection*{(1) Electromagnetic wave equations 电磁波方程}\label{electromagnetic-wave-equations-ux7535ux78c1ux6ce2ux65b9ux7a0b}

\[
\left \{
    \begin{array} {l}
        \left(\dfrac{1}{c^2} \dfrac{\partial^2}{\partial t^2}- \nabla^2\right)\boldsymbol E = \Box \boldsymbol{E} = \boldsymbol 0, \\
        \left(\dfrac{1}{c^2} \dfrac{\partial^2}{\partial t^2}- \nabla^2\right)\boldsymbol B = \Box \boldsymbol{B} = \boldsymbol 0. 
    \end{array}
\right.
\]

Here \(\Box\) or \(\Box^2\) are called the d'Alembert or quabla operator (达朗贝尔算符), and \[\Box =\Box^2 = \dfrac{1}{c^2} \dfrac{\partial^2}{\partial t^2}- \nabla^2.\]

This set of equation is linear, because it is derived from the linear Maxwell's equations.

\subsection*{(2) Navier-Stokes Equations 纳维-斯托克斯方程}\label{navier-stokes-equations-ux7eb3ux7ef4-ux65afux6258ux514bux65afux65b9ux7a0b}

\[\rho \left({\partial \boldsymbol v \over \partial t} + (\boldsymbol v \cdot \nabla ) \boldsymbol v \right) = - \nabla p+\underline{\mu\nabla^2\boldsymbol v}+ \rho \boldsymbol g.\]

When \(t \mapsto -t\), \(\boldsymbol r \mapsto \boldsymbol r\), \(\nabla = \dfrac{\partial}{\partial r} \mapsto \nabla\), \(\boldsymbol v \mapsto -\boldsymbol v\). Consequently to make this equation right, the underlined part (\(\mu\nabla^2\boldsymbol v\)) shall be eliminated to make Euler's equations: \[\rho \left({\partial \boldsymbol v \over \partial t} + (\boldsymbol v \cdot \nabla ) \boldsymbol v \right) = - \nabla p+ \rho \boldsymbol g.\]

\begin{quote}
作业7:了解什么是第5种基本力
\end{quote}
