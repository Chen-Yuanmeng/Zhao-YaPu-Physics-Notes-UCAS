\chapter{2023/12/13}\label{20231213}

\emph{Elon Musk (1971\textasciitilde{} ): First-principle thinking
第一性原理思考:从某一体系的物理本质思考问题}

\emph{越南在明朝之前是中国领土。之后官员搜刮民脂民膏导致当地造反,官员将此事瞒了下来。这是不符合第一性原理的!}

\section{D'Alembert's principle (1743)
达朗贝尔原理}\label{dalmberts-principle-1743-ux8fbeux6717ux8d1dux5c14ux539fux7406}

\subsection*{(1) Contents 内容}\label{contents-ux5185ux5bb9-2}

This principle is hailed as the first principle of theoretical mechanics
(理论力学的第一性原理).

In the system, there are \(s\) particles, and for particle \(i\), its
position vector is
\[\boldsymbol{r}_i = \boldsymbol{r}_i (q_1, q_2, \dots, q_s; t).\]

Particle \(i\) can make real displacements (实位移):
\[\mathrm{d} \boldsymbol{r}_i = \frac{\partial \boldsymbol{r}_i}{\partial t} \mathrm{d}t + \sum_{j = 1}^{s} \frac{\partial \boldsymbol{r}_i}{\partial q_j} \mathrm{d}q_j\]
and we can know its virtual displacement (虚位移):
\[\delta \boldsymbol{r}_i = \sum_{j = 1}^{s} \frac{\partial \boldsymbol{r}_i}{\partial q_j} \delta q_j.\]

From real displacement \(\mathrm{d} \boldsymbol{r}_i\), we can know the
real velocity (实速度) of particle \(i\):
\[\boldsymbol{v}_i = \frac{\mathrm{d} \boldsymbol{r}_i}{\mathrm{d}t} = \frac{\partial \boldsymbol{r}_i}{\partial t} + \sum_{j = 1}^{s} \frac{\partial \boldsymbol{r}_i}{\partial q_j} \frac{\mathrm{d}q_j}{\mathrm{d}t} = \frac{\partial \boldsymbol{r}_i}{\partial t} + \sum_{j = 1}^{s} \frac{\partial \boldsymbol{r}_i}{\partial q_j} \dot{q}_j.\]

From this, we can know that
\[\frac{\partial \boldsymbol{v}_i}{\partial \dot{q}_j} = \sum_{j = 1}^{s} \frac{\partial \boldsymbol{r}_i}{\partial q_j}.\]

Through this, we have reduced a dynamic problem to a static one
(化动为静).

According to the principle of virtual work, we can know
\[\sum_{i = 1}^{n} \boldsymbol{F}_i \cdot \delta \boldsymbol{r}_i = \boldsymbol{0}.\]
According to the Newton's 2nd law of motion, we know that
\[\boldsymbol{F}_i = \frac{\mathrm{d} \boldsymbol{p}_i}{\mathrm{d}t}.\]
From the two equations above, we know that
\[\sum_{i = 1}^{n} \left( \boldsymbol{F}_i - \frac{\mathrm{d} \boldsymbol{p}_i}{\mathrm{d}t} \right) \cdot \delta \boldsymbol{r}_i = 0.\]

This is the \textbf{D'Alembert's principle (达朗贝尔原理)}. It can also
be written in this form:
\[\sum_{i = 1}^{n} \left( \boldsymbol{F}_i - m_i \ddot{\boldsymbol{r}}_i \right) \cdot \delta \boldsymbol{r}_i = 0.\]

\subsection*{(2) Using D'Alembert's principle to derive the
Euler-Lagrange equation
用达朗贝尔原理推导欧拉-拉格朗日方程}\label{using-dalmberts-principle-to-derive-the-euler-lagrange-equation-ux7528ux8fbeux6717ux8d1dux5c14ux539fux7406ux63a8ux5bfcux6b27ux62c9-ux62c9ux683cux6717ux65e5ux65b9ux7a0b}

Generalized force \(Q_j\) is defined as follows:
\[\sum_{i = 1}^{n} \boldsymbol{F}_i \cdot \delta \boldsymbol{r}_i = \sum_{i = 1}^{n} \boldsymbol{F}_i \cdot \left( \sum_{j = 1}^{s} \frac{\partial \boldsymbol{r}_i}{\partial q_j} \delta q_j \right) = \sum_{j = 1}^{s} Q_j \delta q_j = 0. \quad(1)\]

Also, we have \begin{align*}
    \sum_{i = 1}^{n} \boldsymbol{F}_i \cdot \delta \boldsymbol{r}_i & = \sum_{i = 1}^{n} \frac{\mathrm{d} \boldsymbol{p}_i}{\mathrm{d}t} \cdot \left( \sum_{j = 1}^{s} \frac{\partial \boldsymbol{r}_i}{\partial q_j} \delta q_j \right) = \sum_{i = 1}^{n} \sum_{j = 1}^{s}  m_i \frac{\mathrm{d}^2 \boldsymbol{r}}{\mathrm{d}t^2} \cdot \frac{\partial \boldsymbol{r}_i}{\partial q_j} \delta q_j \\
    & = \sum_{i = 1}^{n} \sum_{j = 1}^{s}  m_i \frac{\mathrm{d}}{\mathrm{d}t} \left( \frac{\mathrm{d} \boldsymbol{r}_i}{\mathrm{d}t} \right) \cdot \frac{\partial \boldsymbol{r}_i}{\partial q_j} \delta q_j \\
\end{align*}
\begin{align*}    
    & = \sum_{i = 1}^{n} \sum_{j = 1}^{s}  \left[ \frac{\mathrm{d}}{\mathrm{d}t} \left( m_i \frac{\mathrm{d} \boldsymbol{r}_i}{\mathrm{d}t} \cdot \frac{\partial \boldsymbol{r}_i}{\partial q_j} \right) - m_i \frac{\mathrm{d} \boldsymbol{r}_i}{\mathrm{d}t} \cdot \frac{\mathrm{d}}{\mathrm{d}t} \left( \frac{\partial \boldsymbol{r}_i}{\partial q_j} \right) \right] \delta q_j \\
    & = \sum_{i = 1}^{n} \sum_{j = 1}^{s}  \left[ \frac{\mathrm{d}}{\mathrm{d}t} \left( m_i \boldsymbol{v}_i \cdot \frac{\partial \boldsymbol{v}_i}{\partial \dot{q}_j} \right) - m_i \boldsymbol{v}_i \cdot \frac{\partial}{\partial q_j} \left( \frac{\mathrm{d} \boldsymbol{r}_i}{\mathrm{d}t} \right) \right] \delta q_j \\
    & = \sum_{i = 1}^{n} \sum_{j = 1}^{s}  \left[ \frac{\mathrm{d}}{\mathrm{d}t} \frac{\partial}{\partial \dot{q}_j} \left( {1 \over 2} m_i  {v}_i^2 \right) - \frac{\partial}{\partial q_j} \left( {1 \over 2} m_i  {v}_i^2 \right) \right] \delta q_j \\
    & = \sum_{i = 1}^{n} \sum_{j = 1}^{s}  \left( \frac{\mathrm{d}}{\mathrm{d}t} \frac{\partial T}{\partial \dot{q}_j} - \frac{\partial T}{\partial q_j} \right) \delta q_j \\
    & = 0. \quad(2)
\end{align*}

\emph{编者:精髓就是:凑,就硬凑}

\((1) - (2)\), and we get:
\[\sum_{j = 1}^{s} \left( Q_j - \frac{\mathrm{d}}{\mathrm{d}t} \frac{\partial T}{\partial \dot{q}_j} + \frac{\partial T}{\partial q_j} \right) \delta q_j = 0.\]

Because \(\delta q_j\) ia arbitrary (随意选取的), we need
\[ Q_j - \frac{\mathrm{d}}{\mathrm{d}t} \frac{\partial T}{\partial \dot{q}_j} + \frac{\partial T}{\partial q_j} = 0, \quad(*)\]
where \(Q_j\) is conservative, generalized force (保守的广义力).

Correspondingly, we have a pair of work conjugates (功的共轭):
\[\text{Work conjugate}
\left\{
\begin{array}{ll}
    \boldsymbol{F} & \mathrm{d} \boldsymbol{r} \\
    Q & \delta q
\end{array}
\right.\]

Also we know that \[\frac{\partial V}{\partial \dot{q}_j} = 0\] and
\[Q_j = - \frac{\partial V}{\partial q_j}.\] Substitute these two
equations in \((*)\), and we get
\[- \frac{\partial V}{\partial q_j} - \frac{\mathrm{d}}{\mathrm{d}t} \frac{\partial (T - V)}{\partial \dot{q}_j} + \frac{\partial T}{\partial q_j} = 0\]
\[\frac{\partial (T - V)}{\partial q_j} - \frac{\mathrm{d}}{\mathrm{d}t} \frac{\partial (T - V)}{\partial \dot{q}_j} = 0\]
\[\frac{\partial L}{\partial q_j} - \frac{\mathrm{d}}{\mathrm{d}t} \frac{\partial L}{\partial \dot{q}_j} = 0,\]
where \(L\) is the Lagrangian. This is the Euler-Lagrange equation.

\section{Vibration of a string
弦的振动}\label{vibration-of-a-string-ux5f26ux7684ux632fux52a8}

\emph{编者按:此部分涉及到欧拉-拉格朗日方程的多变量推广。具体的推导可以参考这篇文章:【变分计算2】多变量系统下的变分方程
(\url{https://zhuanlan.zhihu.com/p/357217956})。}

Suppose there is an infinitesimal movement \(u(x, t)\) on a piece of
string whose length is \(L\).

The kinetic energy (动能) of the string is
\[E_k = \int_{0}^{L} {1 \over 2} \mu \left( \frac{\partial u}{\partial t} \right)^2 \mathrm{d}x,\]
where \(\displaystyle \mu = \frac{m}{L}\) is the linear mass density
(质量线密度).

For an infinitesimal length on the string from \(x\) to
\(x + \mathrm{d}x\), when vibrating, its length becomes
\[\mathrm{d}l' = \sqrt{1 + \left( \frac{\partial u}{\partial x} \right)^2} \mathrm{d}x,\]
and the change in length is
\[\mathrm{d}l' - \mathrm{d}x = \left( \sqrt{1 + \left( \frac{\partial u}{\partial x} \right)^2} - 1 \right) \mathrm{d}x \approx \left[ 1 + \frac{1}{2} \left( \frac{\partial u}{\partial x} \right)^2 - 1 \right] = \frac{1}{2} \left( \frac{\partial u}{\partial x} \right)^2 \mathrm{d}x.\]

The potential energy is
\[U = - W = \int T (\mathrm{d}l' - \mathrm{d}x) = \int_{0}^{L} \frac{1}{2} T \left( \frac{\partial u}{\partial x} \right)^2 \mathrm{d}x.\]

The Lagrangian is
\[L = E_k - U = \int_{0}^{L} \frac{1}{2} \left[ \mu \left( \frac{\partial u}{\partial t} \right)^2 - T \left( \frac{\partial u}{\partial x} \right)^2 \right] \mathrm{d}x = \int_{0}^{L} \mathcal{L} \mathrm{d}x,\]
where \(\mathcal{L}\) is the Lagrangian density (拉格朗日量密度).

\(\mathcal{L}\) can be written in this form:
\[\mathcal{L} = \mathcal{L} \left( u, \frac{\partial u}{\partial x}, \frac{\partial u}{\partial t}; x, t \right).\]

Because the Lagrangian is a functional (泛函) of \(u(x, t)\), we can
expand the Euler-Lagrange equation to
\[\frac{\partial}{\partial t} \frac{\partial \mathcal{L}}{\partial (\partial u / \partial t)} + \frac{\partial}{\partial x} \frac{\partial \mathcal{L}}{\partial (\partial u / \partial x)} - \frac{\partial \mathcal{L}}{\partial u} = 0. \quad(*)\]

Here,
\(\displaystyle \mathcal{L} = \frac{1}{2} \left[ \mu \left( \frac{\partial u}{\partial t} \right)^2 - T \left( \frac{\partial u}{\partial x} \right)^2 \right]\)
does not depend on \(u\), and thus \(u\) is a cyclic coordinate
(循环坐标).

Substitute \(\mathcal{L}\) into \((*)\), and we get
\[\mu \frac{\partial^2 u}{\partial t^2} - T \frac{\partial^2 u}{\partial x^2} = 0,\]
\[\left( \frac{1}{c^2} \frac{\partial^2}{\partial t^2} - \frac{\partial^2}{\partial x^2} \right) u = \Box u = 0.\]
Here \(\Box\) is called the \textbf{d'Alembert or quabla operator
(达朗贝尔算符)}:
\[\Box = \frac{1}{c^2} \frac{\partial^2}{\partial t^2}- \nabla^2.\]

\section{Vibration of a membrane
膜的振动}\label{vibration-of-a-membrane-ux819cux7684ux632fux52a8}

Suppose the membrane has an area density (质量面密度) of
\(\varLambda = \rho h\) (where \(h\) is the thickness of the membrane),
and its surface tension coefficient (表面张力系数, surface tension per
unit length) is \(\sigma\).

Similar to the vibration of strings, we have the Lagrangian density
\[\mathcal{L} = \frac{1}{2} \left\{ \varLambda \left( \frac{\partial u}{\partial t} \right)^2 - T \left[ \left( \frac{\partial u}{\partial x} \right)^2 + \left( \frac{\partial u}{\partial y} \right)^2 \right] \right\}\]
and equation of movement
\[\frac{\partial}{\partial t} \frac{\partial \mathcal{L}}{\partial (\partial u / \partial t)} + \frac{\partial}{\partial x} \frac{\partial \mathcal{L}}{\partial (\partial u / \partial x)} + \frac{\partial}{\partial y} \frac{\partial \mathcal{L}}{\partial (\partial u / \partial y)} - \frac{\partial \mathcal{L}}{\partial u} = 0\]
and solution
\[\left( \frac{1}{c^2} \frac{\partial^2}{\partial t^2} - \frac{\partial^2}{\partial x^2} - \frac{\partial^2}{\partial y^2} \right) u = \Box u = 0.\]

\begin{quote}
作业:Arnold《经典力学的数学方法》书中,为什么说所有生物跳的高度是同一量级?
\end{quote}
