\chapter{2023/11/20}\label{20231120}

\section{Something about simple harmonic oscillators (SHO)
简谐振子}\label{something-about-simple-harmonic-oscillators-sho-ux8c10ux632fux5b50}

Some basic things about SHO:

The basic function is \[\ddot x + \omega_0^2 x = 0,\] where
\(\omega_0 = \sqrt{\dfrac{k}{m}}\) and \(x\) is the direction the SHO is
on.

Solve this differential equation and we can get
\[x = A \cos (\omega_0 t + \varphi_0),\] where \(A\) is the amplitude
(振幅) and \(\varphi_0\) is the initial phase of the SHO movement.

Take the time derivative of this and
\[\dot x = - \omega_0 A \sin (\omega_0 t + \varphi_0).\]

The period is
\[T = \dfrac{2 \pi}{\omega_0} = 2 \pi \sqrt{\dfrac{m}{k}}.\]

\subsection*{(1) Relativistic SHO
相对论谐振子}\label{relativistic-sho-ux76f8ux5bf9ux8bbaux8c10ux632fux5b50}

Under relativity, we need to make corrections (修正) to the mass:
\[m = \dfrac{m_0}{\sqrt{1 - \dfrac{v^2}{c^2}}} = \gamma m_0,\] where
\(\gamma\) is the \textbf{Lorentz factor (洛伦兹因子)}.

When \(\dfrac{v}{c} \ll 1\), we expand the Taylor series of \(m\) to
degree 1 and we have
\[m = m_0 \left(1 - \dfrac{v^2}{c^2}\right)^{- 1/2} \approx m_0 \left(1 + \dfrac{v^2}{2c^2}\right),\]
and this partially explains time dilation (钟慢效应).

Let \(v = - \omega_0 A \sin (\omega_0 t + \varphi_0)\), and we have
\[m = m_0 \left[ 1 + \dfrac{ \left( - \omega_0 A \sin (\omega_0 t + \varphi_0) \right) ^2}{2c^2} \right] = m_0 \left[ 1 + \dfrac{ \omega_0^2 A^2 \sin^2 (\omega_0 t + \varphi_0)}{2c^2} \right].\]

Let \(\varepsilon = \dfrac{\dfrac{1}{2} k A^2}{m_0 c^2}\), and obviously
\(\varepsilon \ll 1\). Then we get
\[m = m_0 \left(1 + \varepsilon \sin^2 (\omega_0 t + \varphi_0) \right).\]

Take the time average of \(m\), and we get
\[\overline{m} = m_0 \left(1 + \overline{\sin^2 (\omega_0 t + \varphi_0)} \varepsilon \right) = m_0 \left(1 + \dfrac{1}{2} \varepsilon \right),\]
and thus we have
\[\overline{T} = 2 \pi \sqrt{\dfrac{\overline{m}}{k}} = T_0 \sqrt{1 + \dfrac{1}{2} \varepsilon} \approx T_0 \left( 1 + \dfrac{1}{4} \varepsilon \right).\]

Considering
\(\dfrac{\overline{T} - T_0}{T_0} = \dfrac{1}{4} \varepsilon\) is rather
close to other, more accurate results
\(\left( \dfrac{3}{8} \varepsilon \right)\), we can say that this is
already an acceptable approximation.

\subsection*{(2) Three methods of calculating SHO
计算简谐振子的三种方法}\label{three-methods-of-calculating-sho-ux8ba1ux7b97ux7b80ux8c10ux632fux5b50ux7684ux4e09ux79cdux65b9ux6cd5}

\begin{itemize}
\tightlist{}
\item
  Newtonian Mechanics 牛顿力学

  \[m \ddot x = -kx\] \[\ddot x + {k \over m } x =0 \] \[\ddot x + \omega_0^2 x =0\]

\item
  Conservation of energy 能量守恒方法

  It is clear that the energy of the SHO has this form:
  \[E = \dfrac{1}{2}kx^2 + \dfrac{1}{2} m \dot{x}^2.\]

  Under ideal conditions, \(E\) is conserved, which means that
  \[\dfrac{\mathrm{d} E}{\mathrm{d} t} = kx \dot{x} + m \dot{x} \ddot{x} = \dot{x} \left( kx + m \ddot{x} \right) \equiv 0.\]

  Because \(\dot{x}\) is not conserved, we have \[kx + m \ddot{x} = 0\]
  or \[\ddot x + \omega_0^2 x =0.\]
\item
  Lagrangian mechanics 拉格朗日力学

  \[L=T-V = {1 \over 2}m \dot x^2-{1 \over 2}k x^2\] \[{\partial L \over \partial \dot x} = m \dot x\] \[{\partial L \over \partial x} = -kx\]

  From
  \[{\mathrm d \over \mathrm dt} {\partial L \over \partial \dot x}- {\partial L \over \partial x}=0,\]
  we know that
  \[{\mathrm d(m \dot x) \over \mathrm dt} - (-kx)=0, \]which is
  \[m \ddot x +kx = 0.\] \[\ddot x + \omega_0^2 x =0.\]
\item
  Hamiltonian mechanics 哈密顿力学

  The Hamiltonian (哈密顿量)
  \[H = T + V = {p^2 \over 2m} + {1 \over 2} kx^2.\]

  From the Hamilton canonical equations (哈密顿正则方程) \[\left\{
    \begin{array} {l}
    \dfrac{\partial H}{\partial p_\alpha} = \dot q_\alpha, \\[2ex]
    \dfrac{\partial H}{\partial q_\alpha} = - \dot p_\alpha,
    \end{array}
    \right.\] we have
  \[\dfrac{\partial H}{\partial p} = {p \over m} = \dot{x} \quad \text{and} \quad \dfrac{\partial H}{\partial x} = kx = - \dot{p}.\]

  And thus we have
  \[kx + \dot{p} = kx + {\mathrm{d} \over \mathrm{d}t}\left( m \dot{x} \right) = kx + m \ddot{x} = 0.\]
\end{itemize}

\section{Introduction to Lagrangian Mechanics
拉格朗日力学入门}\label{introduction-to-lagrangian-mechanics-ux62c9ux683cux6717ux65e5ux529bux5b66ux5165ux95e8}

\subsection*{(1) Praise for Lagrangian mechanics
人们对拉格朗日力学的赞誉}\label{praise-for-lagrangian-mechanics-ux4ebaux4eecux5bf9ux62c9ux683cux6717ux65e5ux529bux5b66ux7684ux8d5eux8a89}

\emph{我不想打了!这里罢工!}

\subsection*{(2) Degree of freedom
自由度}\label{degree-of-freedom-ux81eaux7531ux5ea6}

In a mechanical system where there are \(n\) particles, we can determine
the state of the system through \(n\) position vectors
\(\boldsymbol{r}_1, \boldsymbol{r}_2, \dots, \boldsymbol{r}_n\), or a
series of Cartesian coordinates \(u_1, u_2, \dots, u_n\) where
\(u_i = (x_i, y_i, z_i)\).

If, in the system, the number of holonomic constraints (完整约束) is
\(h\), or \[f_i(u_1, u_2, \dots, u_n; t) = 0 \quad(i = 1, 2, ..., h),\]
then we can express \(h\) coordinates using the other \(s = n - h\)
coordinates.

That is to say, we can express the state of the whole system with only
\(s\) coordinates, or, for the sake of convenience, \(s\) independent
parameters. Or to write this in mathematical language,
\[u_i = f(q_1, q_2, \dots, q_s; t) \quad (i = 1, 2, \dots, n).\]

We call these parameters \(q_1, q_2, \dots, q_s\) \textbf{generalized
coordinates (广义坐标)}, and \(s\) \textbf{the degree of freedom (DOF,
自由度)}.

\subsection*{(3) Hamiltonian action
哈密顿作用量}\label{hamiltonian-action-ux54c8ux5bc6ux987fux4f5cux7528ux91cf}

\emph{这里肯定讲不清楚,大家可以参考理论力学课本中有关泛函、哈密顿作用量的有关内容}

The \textbf{Hamiltonian action (哈密顿作用量)} is defined as
\[S = \int_{t_1}^{t_2} L(q_i, \dot{q}_i, t) \mathrm{d}t.\]

This action has the dimension of the following forms:
\[\left[S\right] = \left[E\right] \cdot \left[t\right] = \left[p\right] \cdot \left[q\right] = \left[M\right] \cdot \left[ \theta \right].\]

From this we know that action has 3 conjugate pairs (共轭对): energy and
time (能量和时间), momentum and position (动量和位置), and momentum
torque and angle (动量矩和角度).

\subsection*{(4) 4 principles that the Lagrangian hold
拉格朗日量的4种性质}\label{principles-that-the-lagrangian-hold-ux62c9ux683cux6717ux65e5ux91cfux76844ux79cdux6027ux8d28}

\begin{itemize}
\tightlist{}
\item
  Additivity: When a system can be broken into several separate parts,
  the system's Lagrangian is the sum of the Lagrangian of the parts.
\item
  Not unique
\item
  When the Lagrangian is multiplied by any constant, the differential
  equation of motion still holds.
\item
  If two lagrangians \(L(q, \dot{q}, t)\) and \(L'(q, \dot{q}, t)\) have
  the form
  \[L'(q, \dot{q}, t) = L(q, \dot{q}, t) + {\mathrm{d} \over \mathrm{d} t}f(q, t),\]
  they will imply the same differential equations of motion. For
  example:

  Under Galilean transformation, we have
  \(\boldsymbol{r}' = \boldsymbol{r} - \boldsymbol{V}t\) and \(t' = t\).
  Then, we know \begin{align*}
        L'(\boldsymbol{r}', \dot{\boldsymbol{r}}', t') & = L'(\boldsymbol{r} - \boldsymbol{V}t, \dot{\boldsymbol{r}}' - \boldsymbol{V}, t) \\
        & = L(\boldsymbol{r}, \dot{\boldsymbol{r}}, t) + \dfrac{\partial L}{\partial \boldsymbol{r}} \cdot \left( - \boldsymbol{V} t \right) + \dfrac{\partial L}{\partial \dot{\boldsymbol{r}}} \cdot \left( - \boldsymbol{V} \right) \\
        & = L(\boldsymbol{r}, \dot{\boldsymbol{r}}, t) - {\mathrm{d} \over \mathrm{d} t} \dfrac{\partial L}{\partial \dot{\boldsymbol{r}}} \cdot \left( \boldsymbol{V} t \right) - \dfrac{\partial L}{\partial \dot{\boldsymbol{r}}} \cdot {\mathrm{d} \over \mathrm{d} t} \left( \boldsymbol{V} t \right) \\
        & = L(\boldsymbol{r}, \dot{\boldsymbol{r}}, t) - {\mathrm{d} \over \mathrm{d} t} \left[ \dfrac{\partial L}{\partial \dot{\boldsymbol{r}}} \cdot \left( \boldsymbol{V} t \right) \right].
    \end{align*}
\end{itemize}
