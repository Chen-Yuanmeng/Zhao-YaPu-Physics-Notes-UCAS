\chapter{2023/12/27}\label{20231227}

\section{Dirac equation (1927)
狄拉克方程}\label{dirac-equation-1927-ux72c4ux62c9ux514bux65b9ux7a0b}

\subsection*{(1) The background to proposing the Dirac equation
提出狄拉克方程的背景}\label{the-background-to-proposing-the-dirac-equation-ux63d0ux51faux72c4ux62c9ux514bux65b9ux7a0bux7684ux80ccux666f}

Previously, we have the \textbf{Schrödinger Equation}
\[- { \hbar^2 \over 2m} \nabla^2 \psi + U \psi = \mathrm{i} \hbar {\partial \psi \over \partial t},\]
which can show quantum behavior but not relativistic effects. It
contains the first derivative with respect to time and the second
derivative with respect to space, and thus it does not satisfy Lorentz
invariance.

Later, there was the Klein-Gordon equation
\[\left( \frac{1}{c^2} \frac{\partial^2 }{\partial t^2} - \nabla^2 \right) \psi + \left( \frac{m_0c}{\hbar} \right)^2 \psi = 0,\]
which can show relativistic effects but not quantum behavior.

On this basis, Dirac wanted to put forward an equation containing both
the first derivative with respect to time and space, thus satisfying
both Galilean and Lorentz invariance.

\subsection*{(2) Partial derivation
部分推导}\label{partial-derivation-ux90e8ux5206ux63a8ux5bfc}

We still begin from the momentum and energy operators in quantum
mechanics:
\[\hat{\boldsymbol{p}} = - \mathrm{i} \hbar \nabla \quad \text{and} \quad \hat{E} = \mathrm{i} \hbar \frac{\partial}{\partial t}.\]

Noticing that the deformation (变形) of the energy-momentum relation is
\[E = \sqrt{m_0^2 c^4 + p^2 c^2}, \quad (*)\] and it contains both the
first derivative with respect to time and space, the only thing left to
do for Dirac was to extract the root.

Considering that \(\boldsymbol{p} = (p_x, p_y, p_z)\), we can suppose
the form of the root as
\[\sqrt{a^2 + b^2 + c^2 + d^2} = \alpha_1 a + \alpha_2 b + \alpha_3 c + \beta d.\]
This means that
\[a^2 + b^2 + c^2 + d^2 = \left( \alpha_1 a + \alpha_2 b + \alpha_3 c + \beta d \right)^2,\]
which leads us to \[\left\{
    \begin{array}{l}
        \alpha_i \alpha_j + \alpha_j \alpha_i = 2 \delta_{ij}, \\
        \alpha_i\beta + \beta\alpha_i = 0, \\
        \alpha_i^2 = \beta^2 = 1, \\
    \end{array}
\right.\] where \(i = 1, 2, 3\).

Obviously, for \(\alpha_1, \alpha_2, \alpha_3, \beta\) to have
solutions, these should be matrices. Rewrite these, and we get \[\left\{
    \begin{array}{l}
        \boldsymbol{\alpha}_i \boldsymbol{\alpha}_j + \boldsymbol{\alpha}_j \boldsymbol{\alpha}_i = 2 \delta_{ij} \mathbf{I}, \\
        \boldsymbol{\alpha}_i \boldsymbol{\beta} + \boldsymbol{\beta} \boldsymbol{\alpha}_i = \mathbf{0}, \\
        \boldsymbol{\alpha}_i^2 = \boldsymbol{\beta}^2 = \mathbf{I}. \\
    \end{array}
\right.\]

Here we give the solutions of
\(\boldsymbol{\alpha}_1, \boldsymbol{\alpha}_2, \boldsymbol{\alpha}_3, \boldsymbol{\beta}\)
without proof:
\[\boldsymbol{\beta} = \begin{pmatrix} 1 & 0 & 0 & 0 \\ 0 & 1 & 0 & 0 \\ 0 & 0 & -1 & 0 \\ 0 & 0 & 0 & -1 \end{pmatrix}, \quad \boldsymbol{\alpha}_1 = \begin{pmatrix} 0 & 0 & 0 & 1 \\ 0 & 0 & 1 & 0 \\ 0 & 1 & 0 & 0 \\ 1 & 0 & 0 & 0 \end{pmatrix},\]
\[\boldsymbol{\alpha}_2 = \begin{pmatrix} 0 & 0 & 0 & - \mathrm{i} \\ 0 & 0 & \mathrm{i} & 0 \\ 0 & - \mathrm{i}& 0 & 0 \\ \mathrm{i} & 0 & 0 & 0 \end{pmatrix}, \quad \boldsymbol{\alpha}_3 = \begin{pmatrix} 0 & 0 & 1 & 0 \\ 0 & 0 & 0 & -1 \\ 1 & 0 & 0 & 0 \\ 0 & -1 & 0 & 0 \end{pmatrix}.\]

And the equation \((*)\) becomes
\[\mathrm{i} \hbar \frac{\partial \psi}{\partial t} = \left( - \mathrm{i} \hbar \boldsymbol{\boldsymbol{\alpha}} \cdot \nabla c + \boldsymbol{\beta} mc^2 \right) \psi,\]
where
\(\boldsymbol{\alpha} = (\boldsymbol{\alpha}_1, \boldsymbol{\alpha}_2, \boldsymbol{\alpha}_3)\).
This is the form of Dirac equation. And in natural units, with Feynman
slash notation, we get \[(i\partial \!\!\!/ - m) \psi = 0.\]

This equation can reflect both relativistic and quantum effects, and is
considered one of the most beautiful formulas in theoretical physics.

\section{Hamilton-Jacobi equation
哈密顿-雅克比方程}\label{hamilton-jacobi-equation-ux54c8ux5bc6ux987f-ux96c5ux514bux6bd4ux65b9ux7a0b}

\subsection*{(1) Derivation 推导}\label{derivation-ux63a8ux5bfc-2}

Suppose that we have generalized coordinates \(q_1, q_2, \dots, q_s\).

We already know the Lagrangian \(L(q_j, \dot{q}_j, t)\) and the
Hamiltonian action (哈密顿作用量)
\(S[q(t)] = \int_{t_1}^{t_2} L(q_j, \dot{q}_j, t) \mathrm{d}t\), which
is a functional (泛函) of \(q\).

Maupertuis's principle (莫培督原理) states that the Hamiltonian action
\[S = \int \boldsymbol{p} \cdot \mathrm{d} \boldsymbol{q}.\]

Take the total derivative of \(S\) with respect to time, and we get
\[\frac{\mathrm{d}S}{\mathrm{d}t} = \frac{\partial S}{\partial t} + \sum_{j = 1}^{s} \frac{\partial S}{\partial q_j} \dot{q}_j = \frac{\partial S}{\partial t} + \sum_{j = 1}^{s} p_j \dot{q}_j.\]

Also, because \(\mathrm{d}S = L \mathrm{d}t\) according to the
definition of \(S\), we know that
\[L = \frac{\mathrm{d}S}{\mathrm{d}t} = \frac{\partial S}{\partial t} + \sum_{j = 1}^{s} p_j \dot{q}_j,\]
\[\frac{\partial S}{\partial t} + \sum_{j = 1}^{s} p_j \dot{q}_j - L = 0,\]
\[\frac{\partial S}{\partial t} + H = 0.\]

Also, from above we know that
\(\displaystyle p = \frac{\partial S}{\partial q_j}\), and expressing
the Hamiltonian as
\(\displaystyle H = H \left( q_j, \frac{\partial S}{\partial q_j}, t \right)\),
we get the Hamilton-Jacobi equation:

\[\frac{\partial S}{\partial t} + H \left( q_j, \frac{\partial S}{\partial q_j}, t \right) = 0.\]

\subsection*{(2) Deriving SHO using the Hamilton-Jacobi equation
用哈密顿-雅克比方程推导简谐振子}\label{deriving-sho-using-the-hamilton-jacobi-equation-ux7528ux54c8ux5bc6ux987f-ux96c5ux514bux6bd4ux65b9ux7a0bux63a8ux5bfcux7b80ux8c10ux632fux5b50}

The Hamiltonian of the SHO is
\[H = \frac{p^2}{2m} + \frac{1}{2} kx^2 = \frac{1}{2m} \left( \frac{\partial S}{\partial x} \right)^2 + \frac{1}{2} \omega^2 x^2 = E. \quad (1)\]

From the Hamilton-Jacobi equation, we can get
\[\frac{\partial S}{\partial t} = - H = - E.\]

Adding a function \(W(E, x)\) which does not depend on \(t\), we get
\[S(x, E, t) = - Et + W(E, x). \quad (2)\]

Using \((1)\), we get
\[\frac{\partial S}{\partial x} = \sqrt{2m \left( E - \frac{1}{2}m \omega^2 x^2 \right)};\]
and using \((2)\), we get
\[\frac{\partial S}{\partial x} = \frac{\mathrm{d}W}{\mathrm{d}x},\] and
we get
\[\frac{\mathrm{d}W}{\mathrm{d}x} = \sqrt{2m \left( E - \frac{1}{2}m \omega^2 x^2 \right)},\]
\[\mathrm{d}W = \sqrt{2m \left( E - \frac{1}{2}m \omega^2 x^2 \right)} \mathrm{d}x.\]

Take the partial differential with respect to \(E\), and we get
\begin{align*}
    \frac{\partial W}{\partial E} & = \sqrt{2m} \int \frac{\mathrm{d}x}{2 \sqrt{E - \dfrac{1}{2}m \omega^2 x^2}} \\
    & = \sqrt{\frac{m}{2E}} \int \frac{\mathrm{d}x}{\sqrt{1 - \dfrac{m \omega^2 x^2}{2E}}} \\
    & = \sqrt{\frac{m}{2E}} \sqrt{\frac{2E}{m \omega^2}} \arcsin \left( \sqrt{\frac{m \omega^2}{2E}} x \right) \\
    & = \frac{1}{\omega} \arcsin \left( \sqrt{\frac{m \omega^2}{2E}} x \right).
\end{align*}

According to dimensional analysis, we have
\[\frac{\partial W}{\partial E} - t = \tau,\]
\[\frac{\partial W}{\partial E} = t + \tau = \frac{1}{\omega} \arcsin \left( \sqrt{\frac{m \omega^2}{2E}} x \right),\]
\[x = \sqrt{\frac{2E}{m \omega^2}} \sin \left( \omega t + \varphi_0 \right),\]
where \(\varphi_0 = \omega \tau\).

\section{Poisson bracket
泊松括号}\label{poisson-bracket-ux6ccaux677eux62ecux53f7}

\subsection*{(1) Definition 定义}\label{definition-ux5b9aux4e49-2}

In phase space, we define the Poisson brackets: If we have two functions
\(f(p_j, q_j, t)\) and \(g(p_j, q_j, t)\), then
\[[f, g] = \sum_{j = 1}^{s} \left( \frac{\partial f}{\partial q_j} \frac{\partial g}{\partial p_j} - \frac{\partial f}{\partial p_j} \frac{\partial g}{\partial q_j} \right).\]

Note that many sources also write the Poisson brackets with curly brackets $\{\}$. 

For a random function \(f(p_j, q_j, t)\), we can take the derivative
with respect to time:
\[\frac{\mathrm{d}f}{\mathrm{d}t} = \frac{\partial f}{\partial t} + \sum_{j = 1}^{s} \left( \frac{\partial f}{\partial q_j} \dot{q}_j + \frac{\partial f}{\partial p_j} \dot{p}_j \right).\]

Substitute the Hamiltonian
\(\left\{  \begin{array} {l}  \dfrac{\partial H}{\partial p_j}=\dot q_j \\[1.5ex]  \dfrac{\partial H}{\partial q_j}=-\dot p_j  \end{array} \right.\)
in, and we get
\[\frac{\mathrm{d}f}{\mathrm{d}t} = \frac{\partial f}{\partial t} + \sum_{j = 1}^{s} \left( \frac{\partial f}{\partial q_j} \dfrac{\partial H}{\partial p_j} - \frac{\partial f}{\partial p_j} \dfrac{\partial H}{\partial q_j} \right) = \frac{\partial f}{\partial t} + [f, H].\]

On the occasion that \(f\) does not depend on \(t\), then
\(\dfrac{\partial f}{\partial t} = 0\), and thus
\(\dfrac{\mathrm{d}f}{\mathrm{d}t} = [f, H]\).

\subsection*{(2) Proving Liouville's theorem using
刘维尔定理}\label{proving-liouvilles-theorem-using-ux5218ux7ef4ux5c14ux5b9aux7406}

An equivalent formulation of Liouville's theorem is that
\(\rho(q_j, p_j)\), which is the density of phase points at some point
\((q_j, p_j)\) is conserved, or, in mathematical form,
\[\frac{\mathrm{d} \rho}{\mathrm{d}t} = 0.\]

Take the total derivative of \(\rho\), and we get \begin{align*}
    0 = \frac{\mathrm{d} \rho}{\mathrm{d}t} & = \frac{\partial \rho}{\partial t} + \sum_{j = 1}^{s} \left( \frac{\partial \rho}{\partial q_j} \dot{q}_j + \frac{\partial \rho}{\partial p_j} \dot{p}_j \right) \\
    & = \frac{\partial \rho}{\partial t} + \sum_{j = 1}^{s} \left( \frac{\partial \rho}{\partial q_j} \dfrac{\partial H}{\partial p_j} - \frac{\partial \rho}{\partial p_j} \dfrac{\partial H}{\partial q_j} \right) \\
    & = \frac{\partial \rho}{\partial t} + [\rho, H].
\end{align*}

From this, we know an equivalent formulation of Liouville's theorem:
\[\frac{\partial \rho}{\partial t} + [\rho, H] = 0.\]

\subsection*{(3) Properties 性质}\label{properties-ux6027ux8d28}

There are several properties of the Poisson bracket:

\begin{itemize}
\tightlist{}
\item
  \([f, g] = - [g, f]\);
\item
  \([f, f] = 0\) and \([f, c] = 0\) where \(c = \mathrm{const}\);
\item
  \([f_1 + f_2, g] = [f_1, g] + [f_2, g]\);
\item
  \([f_1 f_2, g] = f_1 [f_2, g] + f_2 [f_1, g]\);
\item
  \(\displaystyle \frac{\partial}{\partial t} [f, g] = [\frac{\partial f}{\partial t}, g] + [f, \frac{\partial g}{\partial t}]\);
\item
  \([q_j, q_j] = [p_j, p_j] = 0\) and \([q_j, p_k] = \delta_{jk}\);
\item
  \([f, [g, h]] + [g, [h, f]] + [h, [f, g]] = 0\).
\end{itemize}

\section{Conclusion and summing up 总结和回顾}\label{conclusion-ux7ed3ux8bba}

\subsection*{(1) ``One''s:}\label{ones}

\begin{itemize}
\tightlist{}
\item
  One center: the beauty of classical mechanics
\item
  One main line: the multiple ways to derive the Simple Harmonic
  Oscillator (SHO)

  \begin{itemize}
\tightlist{}
  \item
    \(\displaystyle \frac{\mathrm{d}E}{\mathrm{d}t} = 0\)
  \item
    the Euler-Lagrange equation
  \item
    the Hamiltonian canonical equations
  \item
    Series expansion
  \item
    Relativistic oscillator (time dilation)
  \item
    Quantum harmonic oscillator
  \item
    Torsional pendulum
  \item
    Physical pendulum
  \end{itemize}
\end{itemize}

\subsection*{(2) ``Two''s:}\label{twos}

\begin{itemize}
\tightlist{}
\item
  Two parts:

  \begin{itemize}
\tightlist{}
  \item
    Newtonian mechanics
  \item
    Analytical mechanics

    \begin{itemize}
\tightlist{}
    \item
      Lagrangian mechanics
    \item
      Hamiltonian mechanics
    \end{itemize}
  \end{itemize}
\end{itemize}

\subsection*{(3) ``Three''s:}\label{threes}

\begin{itemize}
\tightlist{}
\item
  Three geometries:

  \begin{itemize}
\tightlist{}
  \item
    Euclidean geometry
  \item
    Riemann geometry
  \item
    Symplectic geometry
  \end{itemize}
\end{itemize}

\subsection*{(4) ``Four''s:}\label{fours}

\begin{itemize}
\tightlist{}
\item
  Four transformations:

  \begin{itemize}
\tightlist{}
  \item
    Galilean transformation
  \item
    Legendre transformation
  \item
    Lorentz transformation
  \item
    Gauge transformation
  \end{itemize}
\item
  Four basic properties of classical field theory

  \begin{itemize}
\tightlist{}
  \item
    The divergence of the curl of any vector field is equal to zero
  \item
    The curl of the gradient of any scalar field is always the zero
    vector field
  \item
    ``Back Cab''
  \item
    Product operation of Levi-Civita symbols
  \end{itemize}
\item
  Four conserved quantities

  \begin{itemize}
\tightlist{}
  \item
    Energy
  \item
    Momentum
  \item
    Angular momentum
  \item
    Laplace--Runge--Lenz vector
  \end{itemize}
\item
  Four major spaces

  \begin{itemize}
\tightlist{}
  \item
    Euclidean space
  \item
    Configuration space
  \item
    Phase space
  \item
    Minkowski space-time
  \end{itemize}
\item
  Four invariances

  \begin{itemize}
\tightlist{}
  \item
    Galilean invariance
  \item
    Gauge invariance
  \item
    Lorentz invariance
  \item
    Time reversal invariance
  \end{itemize}
\end{itemize}

\subsection*{(5) ``Five''s:}\label{fives}

\begin{itemize}
\tightlist{}
\item
  Five principles

  \begin{itemize}
\tightlist{}
  \item
    The principle of least action
  \item
    Heisenberg's uncertainty principle
  \item
    Virtual work principle
  \item
    D'Alembert's principle
  \item
    Maupertuis's principle
  \end{itemize}
\end{itemize}

\subsection*{(6) ``Six''s:}\label{sixs}

\begin{itemize}
\tightlist{}
\item
  Six major paradoxes

  \begin{itemize}
\tightlist{}
  \item
    Zeno's paradoxes
  \item
    Newton's bucket argument
  \item
    Elevator paradox
  \item
    Twin paradox
  \item
    Bell's spaceship paradox
  \item
    Ehrenfest paradox
  \end{itemize}
\end{itemize}

\subsection*{(7) ``Seven''s:}\label{sevens}

\begin{itemize}
\tightlist{}
\item
  Seven laws

  \begin{itemize}
\tightlist{}
  \item
    Newton's laws of motion (3)
  \item
    Kepler's laws of planetary motion (3)
  \item
    Hooke's law
  \end{itemize}
\end{itemize}

\subsection*{(8) ``Eight''s:}\label{eights}

\begin{itemize}
\tightlist{}
\item
  Eight major groups

  \begin{itemize}
\tightlist{}
  \item
    SO(2)
  \item
    Abel group
  \item
    Lie group
  \item
    Galilean group
  \item
    Poincaré group
  \item
    Lorentz group
  \item
    Space-time translation group
  \item
    Unitary group U(1)
  \end{itemize}
\end{itemize}

\subsection*{(9) ``Nine''s:}\label{nines}

\begin{itemize}
\tightlist{}
\item
  Nine major equations

  \begin{itemize}
\tightlist{}
  \item
    Newton's second law of motion
    \(\displaystyle \boldsymbol{F} = \frac{\mathrm{d}p}{\mathrm{d}t}\)
  \item
    Maxwell's equations
  \item
    Euler--Lagrange equation
  \item
    Hamilton's canonical equation
  \item
    Hamilton--Jacobi equation
  \item
    Navier--Stokes equations
  \item
    Bernoulli's equation
  \item
    Schrödinger equation
  \item
    Klein--Gordon equation
  \end{itemize}
\end{itemize}

\subsection*{(10) ``Ten''s:}\label{tens}

\begin{itemize}
\tightlist{}
\item
  Top ten theorems

  \begin{itemize}
\tightlist{}
  \item
    Noether's theorem
  \item
    Virial theorem
  \item
    Euler's homogeneous function theorem
  \item
    Liouville's theorem
  \item
    Transport theorem in a rotation system
  \item
    Intermediate axis theorem
  \item
    Angular Momentum Theorem
  \item
    König's theorem (柯尼希定理)
  \item
    Gauss's divergence theorem
  \item
    Stokes' theorem
  \end{itemize}
\end{itemize}

\section{Professor Zhao's words
赵爹寄语}\label{professor-zhaos-words-ux8d75ux7239ux5bc4ux8bed}

长江后浪推前浪,世上新人赶旧人。
