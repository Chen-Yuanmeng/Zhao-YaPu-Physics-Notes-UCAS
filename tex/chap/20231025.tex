\chapter{2023/10/25}\label{20231025}

\section{Rigid body rotation with constant angular
velocity \(\boldsymbol \omega_0\)
以恒角速度\(\boldsymbol \omega_0\)转动的刚体}\label{rigid-body-rotation-with-constant-angular-velocity-boldsymbol-omega_0-ux4ee5ux6052ux89d2ux901fux5ea6boldsymbol-omega_0ux8f6cux52a8ux7684ux521aux4f53}

\[\nabla \times \left( \boldsymbol \omega_0 \times \boldsymbol r\right) = 2 \omega_0\]

\section{Field theory 场论}\label{field-theory-ux573aux8bba}

\[\boldsymbol r = x \hat i + y \hat j + z \hat k\]

\[r = \sqrt{x^2 + y^2 + z^2}\]

Common conclusions in field theory:

\begin{itemize}
\tightlist{}
\item
  \(\nabla \cdot \boldsymbol r = 3\)
\item
  \(\nabla \times \boldsymbol r = \boldsymbol 0\)
\item
  \(\nabla r = \hat{\boldsymbol r}\)
\item
  \(\nabla^2 r = \dfrac{2}{r}\)
\item
  \(\nabla \boldsymbol r = \mathbf I\)
\end{itemize}

\section[Deriving Bernoulli's principle 推导伯努利原理]{From Navier-Stokes equations to Bernoulli's principle
从纳维-斯托克斯方程推导伯努利原理}\label{from-navier-stokes-equations-to-bernoullis-principle-ux4eceux7eb3ux7ef4-ux65afux6258ux514bux65afux65b9ux7a0bux63a8ux5bfcux4f2fux52aaux5229ux539fux7406}

The material derivative (物质导数) of \(\boldsymbol v\) is shown as
follows:

\[\boldsymbol a = {\mathrm D \boldsymbol v \over \mathrm Dt} \equiv \underbrace{\partial \boldsymbol v \over \partial t}_\text{Local/Euler acceleration} + \underset{\text{平流加速度 (非线性)}}{\underbrace{\left(\boldsymbol v \cdot \nabla \right) \boldsymbol v}_\text{advective acceleration}}.\]

Euler, 1750: infinitesimal 无穷小量

Why does this form occur?

Suppose we have a macroscopic tensor field (宏观张量场)
\(T = T(t, \boldsymbol r(t))\) with the sense that it depends only on
position and time coordinates:
\[{\mathrm dT \over \mathrm dt} = {\partial T \over \partial t} + {\partial T \over \partial \boldsymbol r} {\mathrm d \boldsymbol r \over \mathrm dt} = {\partial T \over \partial t} + \boldsymbol v {\partial T \over \partial \boldsymbol r} = \left({\partial \over \partial t} + \boldsymbol v \cdot {\partial \over \partial \boldsymbol r} \right) T = \left({\partial \over \partial t} + \boldsymbol v \cdot \nabla \right) T.\]

Using this, we can get: If
\(\boldsymbol v = \boldsymbol v(\boldsymbol r(t), t)\), then
\[\boldsymbol a = {\partial \boldsymbol v \over \partial t} + (\boldsymbol v \cdot \nabla) \boldsymbol v.\]

From the Navier-Stokes equations, we can get:
\[\boldsymbol a = - {1 \over \rho} \nabla p + \nu \nabla^2 \boldsymbol v + \boldsymbol g,\]
which become the Euler equations (fluid dynamics) when \(\nu = 0.\)

By using lamb vector
\[\left(\nabla \times \boldsymbol v \right) \times \boldsymbol v = (\boldsymbol v \cdot \nabla) \boldsymbol v - \nabla \left( {1 \over 2} v^2 \right),\]
we can get
\[{\partial \boldsymbol v \over \partial t} + \left(\nabla \times \boldsymbol v \right) \times \boldsymbol v + \nabla \left( {1 \over 2} v^2 \right) = -{1 \over \rho} \nabla p + \mu \nabla^2 \boldsymbol v + \boldsymbol g.\]

Under the following conditions:

\begin{itemize}
\tightlist{}
\item
  The fluid flows in steady state (定常流动):
  \(\dfrac{\partial \boldsymbol v}{\partial t} = 0\)
\item
  Inviscid fluid (无黏液体): \(\nu = 0\)
\item
  Under conservative force field \(\boldsymbol g = - \nabla (gh)\)
\item
  The fluid is incompressible (液体不可压缩)
\end{itemize}

We can get:
\[\left(\nabla \times \boldsymbol v \right) \times \boldsymbol v + \nabla \left( {1 \over 2} v^2 \right) = - \nabla \left({p \over \rho} \right) - \nabla (gh)\]
\[\left(\nabla \times \boldsymbol v \right) \times \boldsymbol v + \nabla \left({p \over \rho} + {1 \over 2} v^2 + gh \right) = \boldsymbol 0\]
\[{\boldsymbol v \over |\boldsymbol v|} \cdot \left[ \left(\nabla \times \boldsymbol v \right) \times \boldsymbol v + \nabla \left({p \over \rho} + {1 \over 2} v^2 + gh \right) \right] = 0\]
\[{\boldsymbol v \over |\boldsymbol v|} \cdot \left[ \left(\nabla \times \boldsymbol v \right) \times \boldsymbol v \right] + {\boldsymbol v \over |\boldsymbol v|} \cdot \nabla \left({p \over \rho} + {1 \over 2} v^2 + gh \right) = 0\]

Clearly,
\[{\boldsymbol v \over |\boldsymbol v|} \cdot \left[ \left(\nabla \times \boldsymbol v \right) \times \boldsymbol v \right] = 0,\]
because
\(\left(\nabla \times \boldsymbol v \right) \perp \boldsymbol v.\)

Here
\(\displaystyle {\boldsymbol v \over |\boldsymbol v|} \cdot \nabla\) is
a directional derivative (方向导数).

Integral along the streamline (沿流线积分), and we obtain
\[{p \over \rho} + {1 \over 2} v^2 + gh = \mathrm{const},\] which is
called the \textbf{Bernoulli's principle (伯努利原理)}.

Its other forms include: 

\begin{center}
    \begin{tabular}{|c|c|}
        \hline
        \textbf{Form} & \textbf{Name} \\
        \hline
        $\displaystyle \frac{p}{\rho} + \frac{1}{2} v^2 + gh = \mathrm{const}$ & Energy form (per unit mass) \\[1.5ex]
        \hline
        $\displaystyle p + \frac{1}{2} \rho v^2 + \rho g h = \mathrm{const}$ & Pressure form \\[1.5ex]
        \hline
        $\displaystyle \frac{p}{\rho g} + \frac{v^2}{2g} + h = \mathrm{const}$ & Head form (used in Hydraulic engineering) \\[1.5ex]
        \hline
    \end{tabular}
    \captionof{table}{Different forms of Bernoulli's principle}
\end{center}
