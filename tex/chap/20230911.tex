\chapter{2023/9/11}\label{20230911}

\section{Introduction}\label{introduction}

\textbf{2022 Nobel Prize for Physics}: for experiments with entangled photons, establishing the violation of Bell inequalities and pioneering quantum information science

Three people won the prize: \begin{itemize}
\tightlist{}
    \item
    Alain Aspect (France)
    \item
    John Clauser (the U.S.)
    \item
    Anton Zeilinger (Austria)
\end{itemize}

This brings about the problem of \textbf{local realism (局部现实主义)} and \textbf{principle of locality (定域性原理)}, which is believed true in CLASSICAL mechanics or electromagnetics.

\begin{quote}
作业:1. 思考:科学哲学上,``真实''是如何定义的?
\end{quote}

\section{Newtonian Mechanics, 1687 牛顿力学}\label{newtonian-mechanics-1687-ux725bux987fux529bux5b66}

In the \emph{Principia} writes \textbf{the first Principle}: \[\boldsymbol F=\frac{\mathrm d \boldsymbol p} {\mathrm dt}\] and Newton also gives \textbf{the Law of Universal Gravitation}: \[ \boldsymbol F = -\frac{Gm_1m_2}{r^2} \hat {\boldsymbol r}.\]

\section{Some equations 一些公式}\label{some-equations-ux4e00ux4e9bux516cux5f0f}

\begin{itemize}
\tightlist{}
    \item
    Navier-Stokes equations 纳维-斯托克斯方程 \[ {\partial \boldsymbol v \over \partial t} +(\boldsymbol v \cdot \nabla) \boldsymbol v = - {1 \over \rho} \nabla p + \mu \nabla^2 \boldsymbol v + \boldsymbol g\]
    \item
    Euler-Lagrange equation 拉格朗日方程 \[{\partial L \over \partial q_\alpha}-{\mathrm d \over \mathrm dt} {\partial L \over \partial \dot q_\alpha}=0, \quad L=T-V\]
    \item
    Hamilton Canonical Equations 哈密顿正则方程 \[
    \left \{
    \begin{array}{l}
        \dfrac{\partial H}{\partial p_\alpha}=\dot q_\alpha \\[1.5ex]
        \dfrac{\partial H}{\partial q_\alpha}=-\dot p_\alpha
        \end{array}
    \right.
    \]
\end{itemize}

\section{Schrödinger Equation 薛定谔方程}\label{schruxf6dinger-equation-ux859bux5b9aux8c14ux65b9ux7a0b}

\[-{ \hbar^2 \over 2m } \nabla ^2 \psi + U \psi = \mathrm{i} \hbar {\partial \psi \over \partial t}\]

\textbf{Freeman Dyson: Four jokes by Nature}

\begin{itemize}
\tightlist{}
\item
  the square root of minus one that the physicist Erwin Schrödinger put into his wave equation when he invented wave mechanics 被放入薛定谔波函数的-1的平方根
\item
  the precise linearity of quantum mechanics, the fact that the possible states of any physical object form a linear space 量子力学的精确线性,即物理对象的所有可能状态构成线性空间
\item
  the existence of quasi-crystals 拟晶体的存在
\item
  a similarity in behavior between quasi-crystals and the zeros of the Riemann Zeta function 拟晶体和黎曼\(\zeta\)函数在行为上的相似性
\end{itemize}

\textbf{Maxwell's equations 麦克斯韦方程组}

\[
\left \{
    \begin{array} {l} 
        \nabla \cdot \boldsymbol E = \dfrac{\rho}{\varepsilon_0} \\
        \nabla \cdot \boldsymbol B = 0 \\
        \nabla \times \boldsymbol E = -\dfrac{\partial B}{\partial t} \\
        \nabla \times \boldsymbol B = \mu_0(\boldsymbol j+\varepsilon_0 \dfrac{\partial \boldsymbol E}{ \partial t}) \\
    \end{array} 
\right.
\]

\section{General Relativity 广义相对论}\label{general-relativity-ux5e7fux4e49ux76f8ux5bf9ux8bba}

\textbf{爱因斯坦引力场方程} \[G_{\mu v}=R_{\mu v}-{1\over2}g_{\mu v}R=\frac{4\pi G}{c^4}T_{\mu v}\]

\begin{quote}
Matter tells space how to curve. Curvature tells matter how to move.
\end{quote}

\section{Time reversal: A few examples 时间反演性的一些例子}\label{time-reversal-a-few-examples-ux65f6ux95f4ux53cdux6f14ux6027ux7684ux4e00ux4e9bux4f8bux5b50}

When \(t \mapsto -t\): \[\boldsymbol r \mapsto \boldsymbol r\] \[\boldsymbol v = {\mathrm d \boldsymbol r \over \mathrm dt} \mapsto - \boldsymbol v\] \[\boldsymbol a = {\mathrm d \boldsymbol v \over \mathrm dt} \mapsto \boldsymbol a\] \[\cdots\] \[\boldsymbol {U_d} \mapsto -\boldsymbol {U_d} \] \[\boldsymbol j \mapsto - \boldsymbol j\] \[\boldsymbol B \mapsto - \boldsymbol B \]

\subsection*{(1) Lagrangian 拉格朗日量}\label{lagrangian-ux62c9ux683cux6717ux65e5ux91cf}

CPT invariant:

\begin{itemize}
\tightlist{}
\item
  C: Charge Conjugation 电荷共轭变换
\item
  P: Parity 空间反射
\item
  T: Time Reversal 时间反演
\end{itemize}

We can conclude from the Maxwell's equations that the Lagrangian remains invariable under CPT changes.

\subsection*{(2) Schrödinger Equation 薛定谔方程}\label{schruxf6dinger-equation-ux859bux5b9aux8c14ux65b9ux7a0b-1} 

\[\psi(\boldsymbol r, t) \mapsto \psi(\boldsymbol r,-t) \quad (*)\] \[t \mapsto -t\] \[\mathrm i \mapsto -\mathrm i\]

It can be seen that whether the Schrödinger equation remains unchanged under time reversal depends on \((*)\).

\subsection*{(3) Euler-Lagrange equation 拉格朗日方程}\label{euler-lagrange-equation-ux62c9ux683cux6717ux65e5ux65b9ux7a0b}

The equation is \[{\mathrm d \over \mathrm dt} {\partial L \over \partial \dot q_\alpha}- {\partial L \over \partial q_\alpha}=0, \quad L= T - V,\] in which \(q_\alpha\) is the \textbf{generalized coordinate} (广义坐标), and \(\dot q_\alpha\) is its \textbf{generalized velocity} (广义速度).

\begin{quote}
作业:2. 推导哈密顿正则方程 \[
\left \{
    \begin{array} {l}
        \dfrac{\partial H}{\partial p_\alpha}=\dot q_\alpha \\[1.5ex]
        \dfrac{\partial H}{\partial q_\alpha}=-\dot p_\alpha
    \end{array}
\right.
\] 的时间反演不变性。
\end{quote}

\section{Hamilton-Jacobian equation 哈密顿-雅可比方程}\label{hamilton-jacobian-equation-ux54c8ux5bc6ux987f-ux96c5ux53efux6bd4ux65b9ux7a0b}

\[S= \int^{t_1}_{t_2} L \mathrm dt\] \[ {\partial S \over \partial t}+H =0\]
