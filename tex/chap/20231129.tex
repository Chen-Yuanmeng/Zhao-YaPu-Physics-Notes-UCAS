\chapter{2023/11/29}\label{20231129}

\section{Derivation of the Euler-Lagrange equation using the
principle of least action
用最小作用量原理推导欧拉-拉格朗日方程}\label{derivation-of-the-euler-lagrange-equation-using-the-principle-of-least-action-ux7528ux6700ux5c0fux4f5cux7528ux91cfux539fux7406ux63a8ux5bfcux6b27ux62c9-ux62c9ux683cux6717ux65e5ux65b9ux7a0b}

According to the definition of action, we have
\[S = \int_{t_1}^{t_2} L(q, \dot{q}, t) \mathrm{d}t.\]

According to the principle of least action, we have \[\delta S = 0,\]
and under isochronic variation (等时变分, \(\delta t = 0\)) we get
\begin{align*}
    \delta S & = \delta \int_{t_1}^{t_2} L(q, \dot{q}, t) \mathrm{d}t \\
    & = \int_{t_1}^{t_2} \left( {\partial L \over \partial q} \delta q + {\partial L \over \partial \dot{q}} \delta \dot{q} \right) \mathrm{d}t.
\end{align*}

Because in this condition, we have
\(\delta \dot{q} = \dfrac{\mathrm{d}}{\mathrm{d}t} \delta q\). Perform
integration by parts, and we have
\[\delta S = {\partial L \over \partial \dot{q}} \delta q \Bigg|_{t_1}^{t_2} + \int_{t_1}^{t_2} \left( {\partial L \over \partial q} \delta q - \dfrac{\mathrm{d}}{\mathrm{d}t} {\partial L \over \partial \dot{q}} \delta q \right) \mathrm{d}t = 0.\]

Because
\(\displaystyle {\partial L \over \partial \dot{q}} \delta q \Bigg|_{t_1}^{t_2} = 0\),
\(\displaystyle \int_{t_1}^{t_2} \left( {\partial L \over \partial q} \delta q - \dfrac{\mathrm{d}}{\mathrm{d}t} {\partial L \over \partial \dot{q}} \delta q \right) \mathrm{d}t\)
should be \(0\) for any \(\delta q\). Consequently, for any coordinate,
there should hold
\[{\partial L \over \partial q} - \dfrac{\mathrm{d}}{\mathrm{d}t} {\partial L \over \partial \dot{q}} = 0,\]
which is the Euler-Lagrange equations.

\section{Cyclic/Ignorable coordinates
循环/可遗坐标}\label{cyclicignorable-coordinates-ux5faaux73afux53efux9057ux5750ux6807}

If the Lagrangian \(L\) does not depend on some coordinate \(q_\alpha\),
it follows immediately from the Euler--Lagrange equations that
\[{\mathrm{d} p_\alpha \over \mathrm{d}t} = {\mathrm{d} \over \mathrm{d}t}{\partial L \over \partial \dot{q}_\alpha} = {\partial L \over \partial q_\alpha} = 0.\]

Integrate this, and we know that \[p_\alpha = \mathrm{const}.\]

In this case, we call \(q_\alpha\) \textbf{cyclic/ignorable coordinates
(循环/可遗坐标)}, and the corresponding momentum is conserved.

In this case, generalized force is
\[{\partial L \over \partial q_\alpha} = - {\partial U \over \partial q_\alpha},\]
and generalized momentum is
\[{\partial L \over \partial \dot{q}_\alpha} = {\partial T \over \partial \dot{q}_\alpha}.\]

Two examples of cyclic coordinates:

\begin{itemize}
\tightlist{}
\item
  Kepler's problem 开普勒问题

  Suppose we have a central force field that follows the inverse square
  law (平方反比有心力场). In polar coordinates \(r\) and \(\theta\), we
  have the Lagrangian
  \[L = {1 \over 2} m \left( \dot{x}^2 + r^2 \dot{\theta}^2 \right) - \left( - {GMm \over r} \right).\]

  This Lagrangian does not include \(\theta\), and \(\theta\) is a
  cyclic/ignorable coordinate here.

  Thus, we now that the generalized momentum
  \[p_\theta = {\partial L \over \partial \dot{\theta}} = mr^2 \dot{\theta}\]
\item
  Motion of a particle in a gravitational field 重力场中粒子的运动

  We have the Lagrangian
  \[L = {1 \over 2} m \left( \dot{x}^2 + \dot{y}^2 + \dot{z}^2 \right) - mgz,\]
  which doesn't contain \(x\) and \(y\).

  From this we know that
  \[p_x = {\partial L \over \partial \dot{x}} = m \dot{x} = \mathrm{const}\]
  and
  \[p_y = {\partial L \over \partial \dot{y}} = m \dot{y} = \mathrm{const}.\]
\end{itemize}

\section{Metric tensors
度规张量}\label{metric-tensors-ux5ea6ux89c4ux5f20ux91cf-2}

\subsection*{(1) Minkowski spacetime
闵可夫斯基时空}\label{minkowski-spacetime-ux95f5ux53efux592bux65afux57faux65f6ux7a7a-1}

In Minkowski spacetime, the space-time interval can be written as
\begin{align*}
    \mathrm{d} s^2 & = g_{\mu\nu} \mathrm{d}x^{\mu} \mathrm{d}x^{\nu} = c^2 \mathrm{d}t^2 - \mathrm{d} x^2 - \mathrm{d} y^2 - \mathrm{d} z^2 \\
    & = c^2 \mathrm{d}t^2 \left[ 1 - \frac{1}{c^2} \left( \frac{\mathrm{d}x}{\mathrm{d}t} \right)^2 - \frac{1}{c^2} \left( \frac{\mathrm{d}y}{\mathrm{d}t} \right)^2 - \frac{1}{c^2} \left( \frac{\mathrm{d}z}{\mathrm{d}t} \right)^2 \right] \\
    & = c^2 \mathrm{d}t^2 \left( 1 - {v^2 \over c^2} \right) = c^2 \mathrm{d} \tau^2,
\end{align*}

where \(\mathrm{d} s^2\) stands for \((\mathrm{d} s)^2\).

From this we can see \[\mathbf{g} = \begin{pmatrix}
    1 & 0 & 0 & 0 \\
    0 & -1 & 0 & 0 \\
    0 & 0 & -1 & 0 \\
    0 & 0 & 0 & -1
\end{pmatrix}.\]

\subsection*{(2) Schwarzschild metric
史瓦西度规}\label{schwarzschild-metric-ux53f2ux74e6ux897fux5ea6ux89c4}

\emph{这个解是史瓦西在前线、在战壕里看爱因斯坦的文章学相对论时解出来的。}

The Schwarzschild metric is an exact solution to the Einstein field
equations (爱因斯坦场方程) that describes the gravitational field
outside a spherical mass:
\[R_{\mu\nu} - {1 \over 2} g_{\mu\nu} R = {8 \pi G \over c^4} T_{\mu\nu}.\]

The solution is
\[\mathrm{d}s^2 = \dfrac{1}{1 - \dfrac{r_s}{r}} \mathrm{d}r^2 + r^2 \mathrm{d} \theta^2 + r^2 \sin^2 \theta \mathrm{d} \varphi^2 - \left( 1 - \dfrac{r_s}{r} \right) c^2 \mathrm{d}t^2,\]
where \(r_s\) is the Schwarzschild radius. It meets
\(\displaystyle {1 \over 2} m v^2 - {GMm \over r_s} = 0\) at velocity \(v = c\). The
Schwarzschild radius is \(\displaystyle r_s = {2GM \over c^2}\).

Write the Schwarzschild metric in matrix form and we have
\[\mathbf{g} = (g_{\mu\nu}) = \begin{pmatrix}
    \dfrac{1}{1 - \dfrac{r_s}{r}} & 0 & 0 & 0 \\
    0 & r^2 & 0 & 0 \\
    0 & 0 & r^2 \sin^2 \theta & 0 \\
    0 & 0 & 0 & - \left( 1 - \dfrac{r_s}{r} \right) \\
\end{pmatrix}.\]

\subsection*{(3) Spherical coordinates
球坐标}\label{spherical-coordinates-ux7403ux5750ux6807}

In this situation we have
\(\boldsymbol{r} = (r \sin \theta \cos \varphi, r \sin \theta \sin \varphi, r \cos \theta)\),
or
\[\mathrm{d} s^2 = \mathrm{d}r^2 + r^2 \mathrm{d} \theta^2 + r^2 \sin^2 \theta \mathrm{d} \varphi^2 = g_{\mu\nu} \mathrm{d}x^{\mu} \mathrm{d}x^{\nu}.\]

Thus, we have
\[\frac{\partial \boldsymbol{r}}{\partial r} = (\sin \theta \cos \varphi, \sin \theta \sin \varphi, \cos \theta),\]
\[\frac{\partial \boldsymbol{r}}{\partial \theta} = (r \cos \theta \cos \varphi, r \cos \theta \sin \varphi, - r \sin \theta),\]
\[\frac{\partial \boldsymbol{r}}{\partial \varphi} = (- r \sin \theta \sin \varphi, r \sin \theta \cos \varphi, 0).\]

From this we know
\[g_{rr} = (\sin \theta \cos \varphi)^2 + (\sin \theta \sin \varphi)^2 + (\cos \theta)^2 = 1,\]
\[g_{\theta \theta} = (r \cos \theta \cos \varphi)^2 + (r \cos \theta \sin \varphi)^2 + (- r \sin \theta)^2 = r^2,\]
\[g_{\varphi \varphi} = r^2 \sin^2 \theta.\]

\[\mathbf{g} = (g_{\mu\nu}) = \begin{pmatrix}
    1 & 0 & 0 \\
    0 & r^2 & 0 \\
    0 & 0 & r^2 \sin^2 \theta
\end{pmatrix}.\]

\section{Mechanical--electrical analogy
力-电类比}\label{mechanicalelectrical-analogy-ux529b-ux7535ux7c7bux6bd4}

\begin{center}
    \includegraphics[height=120pt]{assets/Mechanical–electrical_analogy.png}
    \captionof{figure}{Mechanical--electrical analogy}
\end{center}

We put a form here to fully represent this.

\begin{center}
    \begin{tabular}{|c|c|c|c|}
        \hline
        Quantities \& Functions & Mechanical System & Electrical System & Analogy \\
        \hline
        Kinetic energy & $\displaystyle T = {1 \over 2} m \dot{x}^2$ & $\displaystyle T = {1 \over 2} L \dot{Q}^2$ & $\displaystyle m \Leftrightarrow L$ \\
        Potential energy & $\displaystyle V = {1 \over 2} kx^2$ & $\displaystyle V = {Q^2 \over 2C}$ & $\displaystyle k \Leftrightarrow {1 \over C}$ \\
        Lagrangian & $\displaystyle \mathcal{L}(x, \dot{x}) = {1 \over 2} m \dot{x}^2 - {1 \over 2} kx^2$ & $\displaystyle \mathcal{L}(Q, \dot{Q}) = {1 \over 2} L \dot{Q}^2 - {Q^2 \over 2C}$ & $\displaystyle x \Leftrightarrow Q$ \\ \hline
        Dissipation function & $\displaystyle \mathcal{R} = {1 \over 2} \eta \dot{x}^2$ & $\displaystyle \mathcal{R} = {1 \over 2} R \dot{Q}^2 = {1 \over 2} RI^2$ & $\displaystyle \eta \Leftrightarrow R$ \\
        & & & $\displaystyle \dot{x} \Leftrightarrow I = \dot{Q}$ \\
        \hline
        & $\displaystyle m \ddot{x} + \eta \dot{x} + kx = 0$ & $\displaystyle L \ddot{Q} + R \dot{Q} + {Q \over C} = 0$ & \\
        Kinetic equations & $\displaystyle \ddot{x} + c \dot{x} + \omega_0^2 x = 0$  &  $\displaystyle \ddot{Q} + \mu \dot{Q} + \omega_0^2 Q = 0$ & $\displaystyle c \Leftrightarrow \mu$ \\
        & $\displaystyle \left( c = {\eta \over m}, \omega_0 = \sqrt{k \over m} \right)$ & $\displaystyle \left( \mu = {R \over L}, \omega_0 = \sqrt{1 \over LC} \right)$ & \\
        \hline
        & & & $\displaystyle \eta \Leftrightarrow R$ \\
        Quality factors & $\displaystyle \mathcal{Q} = \frac{\sqrt{km}}{\mu}$ & $\displaystyle \mathcal{Q} = {1 \over R} \sqrt{L \over C}$ & $\displaystyle m \Leftrightarrow L$ \\
        & & & $\displaystyle k \Leftrightarrow {1 \over C}$ \\
        \hline
        \end{tabular}
        \captionof{table}{Analogies between Mechanical and Electrical Systems}
\end{center}
